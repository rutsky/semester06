% golden_section_search.tex
% Report on using golden section search algorithm for finding one dimensional special type function minimum.
% Vladimir Rutsky <altsysrq@gmail.com>
% 24.03.2009

\documentclass[10pt,a4paper,titlepage]{article}

% Spaces after commas.
\frenchspacing
% Minimal carrying number of characters,
\righthyphenmin=2

% Encoding support.
\usepackage{ucs}
\usepackage[utf8x]{inputenc}
\usepackage[T2A]{fontenc}
\usepackage[russian]{babel}

\usepackage{amsmath, amsthm, amssymb}

% From K.V.Voroncov Latex in samples, 2005.
\textheight=24cm   % Text height.
\textwidth=16cm    % Text width
\oddsidemargin=0pt % Left side indention.
\topmargin=-1.5cm  % Top side indention.
\parindent=24pt    % Paragraph indent.
\parskip=0pt       % Distance between paragraphs.
\tolerance=2000
%\flushbottom       % Page height aligning.

% Algorithms stuff.
\usepackage{algorithm}
\usepackage{algorithmic}

\newcommand{\INPUT}{\textbf{Вход:~}}
\newcommand{\OUTPUT}{\textbf{Выход:~}}
\floatname{algorithm}{Алгоритм}
\renewcommand{\algorithmicrepeat}{\textbf{do}}
\renewcommand{\algorithmicuntil}{\textbf{while}}

% Drawing figures.
\usepackage{subfig}

% Drawing graphs.
\usepackage{tikz}
\usetikzlibrary{trees,positioning,arrows}

% For listings.
\usepackage{listings}

% Transpose symbol in math mode.
\newcommand\T{^\mathrm{\textsf{\tiny{T}}}}

% Source code listings.
\renewcommand{\lstlistingname}{Исходный код}


\title{Одномерная минимизация функции}
\author{Владимир Руцкий, 3057/2}
%\date{}

\begin{document}

\maketitle
\thispagestyle{empty}

% Content.
\section{Постановка задачи}
Требуется найти с наперёд заданной точностью минимум одномерной функции $f(x)$ на заданном отрезке $[a, b]$:
$$ \min f(x), \quad x \in [a, b], $$
используя \textit{метод золотого сечения}.

Дана конкретная функция $\frac{\cos x}{x^2}
,$ и задан отрезок $[7, 11].$

\section{Исследование применимости метода}
Алгоритмы поиска минимума одномерной функции $f(x)$ на отрезке $[a, b]$, 
использующие только значения функции в некоторых точках 
(исследуемый метод золотого сечения в их числе), применимы, только
если заданная функция является \textit{унимодальной}, а именно
$$ \exists ! \, x^* \in [a, b]: \quad
\begin{cases}
  \forall x_1, x_2 \in [a,b]: \quad x^* \leqslant x_1 \leqslant x_2 & \Rightarrow f(x^*) \leqslant f(x_1) \leqslant f(x_2) \\
  \forall x_1, x_2 \in [a,b]: \quad x^* \geqslant x_1 \geqslant x_2 & \Rightarrow f(x^*) \geqslant f(x_1) \geqslant f(x_2) \\
\end{cases}.
$$

\section{Описание алгоритма}
Общая идея поиска минимума унимодальной функции на отрезке заключается в том, 
что отрезок с минимумом можно разбить на части и гарантировано определить в какой части находится минимум,
тем самым позволив сузить область для поиска минимума.

В случае метода золотого сечения отрезок, на котором имеется минимум, $[a,b]$ делится в отношении 
\textit{золотой пропорции}.
\textit{Золотая пропорция}~--- это деление непрерывной величины на части в таком отношении, 
что большая часть относится к меньшей, как большая ко всей величине.

$x \in [a,b]$ делит $[a,b]$ золотым сечением, если 
$$
\begin{cases}
  x - a > b - x &  \frac{x - a}{b - x} = \frac{b - a}{x - a} = \varphi \\
  x - a < b - x &  \frac{b - x}{x - a} = \frac{b - a}{b - x} = \varphi \\
\end{cases},
$$
т.е. для любого непустого отрезка найдуться две точки, делящие его в отношении золотой пропорции.

Величина отношения $\varphi$ определена единственным образом, и равна $\frac{\sqrt{5} + 1}{2}.$

Поиск минимума унимодальной функции методом золотого сечения можно описать алгоритмом~\ref{FindMinimumByGoldenSectionSearch}:
\begin{algorithm}[H]
\caption{FindMinimumByGoldenSectionSearch. Поиск минимума унимодальной функции.}
\label{FindMinimumByGoldenSectionSearch}
\INPUT Исследуемая функция $f$; отрезок $[a,b]$, содержащий минимум; точность $\varepsilon$, с которой требуется найти минимум. \\
\OUTPUT Точка $x^*$~--- аргумент функции в которой достигается минимум с точностью $\varepsilon$.

\begin{algorithmic}
\STATE \COMMENT { Инициализируем первоначальное деление отрезка $[a,b]$ на три отрезка золотым сечением. }
\STATE $\alpha$ := $\frac{1}{\varphi}$ \COMMENT { с данной величиной далее будет удобно работать }
\STATE $i$ := 0 \COMMENT { счетчик итераций }
\STATE \COMMENT { $[a_i,b_i]$~--- отрезок содержащий минимум на $i$-м шаге. }
\STATE $a_i$ := $a$
\STATE $b_i$ := $b$
\STATE \COMMENT { $\lambda_i < \mu_i$~--- точки делящие отрезок на $i$-м шаге золотым сечением. }
\STATE $\lambda_i$ := $a_i + (1 - \alpha)(b_i - a_i)$
\STATE $\mu_i$ := $a_i + \alpha(b_i - a_i)$
\STATE \COMMENT { Подразбиваем отрезок пока его длина не станет меньше необходимой точности. }
\WHILE { $|a_i - b_i| \geqslant \varepsilon$ }
  \IF { $f(\lambda_i) \leqslant f(\mu_i)$ }
    \STATE \COMMENT { Минимум лежит на $[a_i,\mu_i]$, подразбиваем этот отрезок и продолжаем работу алгоритма. }
    \STATE $a_{i+1}$ := $a_i$
    \STATE $b_{i+1}$ := $\mu_i$
    \STATE \COMMENT { Из-за особенностей золотого сечения необходимо перевычислять лишь одну новую точку. }
    \STATE $\mu_{i+1}$ := $\lambda_i$
    \STATE $\lambda_{i+1}$ := $a_{i+1} + (1 - \alpha)(b_{i+1} - a_{i+1})$
  \ELSE
    \STATE \COMMENT { Минимум лежит на $[\lambda_i,b_i]$. }
    \STATE $a_{i+1}$ := $\lambda_i$
    \STATE $b_{i+1}$ := $b_i$
    \STATE $\lambda_{i+1}$ := $\mu_i$
    \STATE $\mu_{i+1}$ := $a_{i+1} + \alpha(b_{i+1} - a_{i+1})$
  \ENDIF
  \STATE $i$ := $i + 1$
\ENDWHILE
\STATE \COMMENT { Все точки на результирующем отрезке равны точке минимума с точностью $\varepsilon$, %
  выбираем точку с меньшим значением функции из $\lambda_i$ и $\mu_i$.  }
\IF { $f(\lambda_i) \leqslant f(\mu_i)$ }
  \RETURN $\lambda_i$
\ELSE
  \RETURN $\mu_i$
\ENDIF
\end{algorithmic}
\end{algorithm}

Приведённый алгоритм~\ref{FindMinimumByGoldenSectionSearch} может быть легко оптимизирован, 
если сохранять вычисленные значения функции $f$ в точках $\alpha_i$ и $\mu_i$, 
таким образом функция может быть вычислена значительно меньшее число раз. 
Для простоты эта оптимизация не приводится в алгоритме.

\section{Код программы}
\lstset{language=Octave, caption=Решение задачи поиска минимума унимодальной функции,%
label=main-source-code, basicstyle=\footnotesize,%
numbers=left, numberstyle=\footnotesize, numbersep=5pt, frame=single, breaklines=true, breakatwhitespace=false,%
inputencoding=utf8x}
\lstinputlisting{data/golden_section_search.m}

\section{Результаты решения}

\section{Возможные дополнительные исследования}
\section{Обоснование достоверности полученного результата}

\end{document}
