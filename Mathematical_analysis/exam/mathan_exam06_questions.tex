\documentclass[a4paper,10pt]{article}

% Encoding support.
\usepackage{ucs}
\usepackage[utf8x]{inputenc}
\usepackage[T2A]{fontenc}
\usepackage[russian]{babel}

% Spaces after commas.
\frenchspacing
% Minimal carrying number of characters,
\righthyphenmin=2

\usepackage{amsmath, amsthm, amssymb}

% From K.V.Voroncov Latex in samples, 2005.
\textheight=24cm   % text height
\textwidth=16cm    % text width.
\oddsidemargin=0pt % left side indention
\topmargin=-1.5cm  % top side indention.
\parindent=24pt    % paragraph indent
\parskip=0pt       % distance between paragraphs.
\tolerance=2000
%\flushbottom       % page height aligning
%\hoffset=0cm
\pagestyle{empty}  % withour numeration

%\renewcommand{\baselinestretch}{0.5} % lowering lines interval

\begin{document}

\section*{Вопросы к~экзамену по~математическому анализу \\%
Физ.-мех., мех.~поток и ФУИТ, III~курс, 2 сем. 2008/09 уч.~г.\\%
Лектор~--- доц.~Басов~А.\,В.}

\begin{enumerate}
  \setlength{\itemsep}{-0.1mm} % отступ в перечислении --- выглядит страшно

  \item Теорема о среднем для голоморфных функций.
  \item Локальный принцип максимума.
  \item Глобальный принцип максимума.
        Следствие.
  \item Лемма Шварца о функциях, голоморфных в круге.
  \item Гармонические функции.
        Гармоничность голоморфных функций.
        Связь между вещественными гармоническими и голоморфными функциями.
        Гармоничность функции $y = \ln |z|$.
  \item Связь между вещественными гармоническими и голоморфными функциями в окрестности произвольной точки.
        Пример гармонической функции, которая не является вещественной частью никакой голоморфной функции.
        Связь между вещественными гармоническими и голоморфными функциями в односвязной области.
  \item Теорема о среднем для гармонических функций.
  \item Интегральное представление функции, гармонической в круге.
        Формула Пуассона.
        Свойства ядра Пуассона.
  \item Постановка задачи Дирихле для круга.
        Лемма о гармонической функции, равной нулю на окружности.
        Единственность решения задачи Дирихле.
  \item Существование решения задачи Дирихле для круга.
  \item Определение гармонических функций с помощью теоремы о среднем.
  \item Свойства голоморфных отображений с отличной от нуля производной.
        Предложение о виде сохраняющего углы однородного линейного отображения комплексной плоскости на себя.
  \item Предложение о виде сохраняющего углы непрерывно дифференцируемого отображения области комплексной плоскости.
  \item Локальные свойства голоморфного отображения с обращающейся в ноль производной.
  \item Открытые отображения метрических пространств.
        Непрерывность отображения, обратного к открытому.
  \item Предложение об открытости голоморфных отображений.
        Однолистные отображения.
        Изоморфизмы областей комплексной плоскости.
        Автоморфизмы.
  \item Пример неизоморфных гомеоморфных областей.
        Представление произвольного изоморфизма областей с помощью фиксированного.
        Абстрактные группы.
        Группы автоморфизмов.
  \item Группа автоморфизмов комплексной плоскости.
        Транзитивность действия группы автоморфизмов.
  \item Сфера Римана.
        Лемма о подгруппе группы автоморфизмов, действующей транзитивно.
  \item Продолжение дробно-линейных отображений на сферу Римана.
        Группа автоморфизмов сферы Римана.
  \item Круговое свойство дробно-линейных преобразований.
        Изоморфизм верхней полуплоскости и единичного круга.
  \item Нахождение подгруппы группы автоморфизмов сферы Римана, переводящих верхнюю полуплоскость на себя.
  \item Транзитивность действия подгруппы группы автоморфизмов сферы Римана, переводящих верхнюю полуплоскость на себя.
        Предложение об изоморфизме между группами автоморфизмов верхней полуплоскости и единичного круга.
  \item Предложение об автоморфизмах единичного круга, переводящих ноль в ноль.
        Следствие об автоморфизмах верхней полуплоскости, переводящих в.
        Группа автоморфизмов верхней полуплоскости.
        Нахождение дробно-линейных преобразований, переводящих единичную окружность на себя.
  \item .Нахождение дробно-линейных преобразований, переводящих внутренность единичного круга на себя.
        Группа автоморфизмов единичного круга.
        Основная теорема о конформных отображениях (без доказательства).
  \item Постановка задачи Коши для системы дифференциальных уравнений.
        Теоремы существования, единственности и зависимости от параметров решений систем (без доказательства).
  \item Автономные системы дифференциальных уравнений.
        Графики решений и фазовые траектории, связь между ними.
        Фазовый поток.
  \item Предложение о добавлении константы к аргументу решения.
        Свойство фазовых траекторий двух решений.
  \item Оператор эволюции и его свойства.
  \item Три типа фазовых траекторий автономной системы.
  \item Фазовые портреты автономных систем.
        Локальный фазовый портрет вблизи неособой точки.
        Система линейного приближения.
  \item Устойчивость и асимптотическая устойчивость по Ляпунову.
        Линейные системы первого порядка.
  \item Преобразование линейной системы второго порядка при переходе к новому базису в фазовом пространстве.
        Построение решений при наличии двух линейно независимых собственных векторов у матрицы системы.
  \item Построение вещественных решений линейной системы второго порядка в случае комплексных собственных чисел у матрицы системы.
  \item Построение решений линейной системы второго порядка в случае одномерного пространства собственных векторов у матрицы системы.
  \item Построение фазовых портретов невырожденных линейных систем: узел и седло.
  \item Построение фазовых портретов невырожденных линейных систем: звездный (дикритический) узел.
  \item Построение фазовых портретов невырожденных линейных систем: центр и фокус.
  \item Построение фазовых портретов невырожденных линейных систем: вырожденный узел.
  \item Построение фазовых портретов вырожденных линейных систем второго порядка.
  \item Зависимость фазового портрета от следа и определителя матрицы системы.
  \item Возмущение системы дифференциальных уравнений.
        Примеры различного поведения точки покоя при возмущении системы.
        Теорема о поведении точки покоя при возмущении системы.
  \item Топологическая эквивалентность систем.
        Примеры.
        Теорема Гробмана--Хартмана о структурной устойчивости (без доказательства).
        Понятие бифуркации.
  \item Неподвижные точки, сжимающие отображения.
        Теорема Банаха о сжимающих отображениях.
  \item Предложение об устойчивости неподвижной точки.
  \item Предложение о полноте пространства функций, непрерывных на отрезке.
        Полнота пространства непрерывных вектор-функций на отрезке.
  \item Теорема Пикара о существовании и единственности решения задачи Коши для системы дифференциальных уравнений.
        Предложение о непрерывной зависимости решения от параметра.
  \item Теорема о неявном отображении.
        Замечания.

\end{enumerate}

\end{document}
