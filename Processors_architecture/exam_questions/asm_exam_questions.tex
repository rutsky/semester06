\documentclass[a4paper,10pt]{article}

% Encoding support.
\usepackage{ucs}
\usepackage[utf8x]{inputenc}
\usepackage[T2A]{fontenc}
\usepackage[russian]{babel}

% Spaces after commas.
\frenchspacing
% Minimal carrying number of characters,
\righthyphenmin=2

\usepackage{amsmath, amsthm, amssymb}

% From K.V.Voroncov Latex in samples, 2005.
\textheight=24cm   % text height
\textwidth=16cm    % text width.
\oddsidemargin=0pt % left side indention
\topmargin=-1.5cm  % top side indention.
\parindent=24pt    % paragraph indent
\parskip=0pt       % distance between paragraphs.
\tolerance=2000
%\flushbottom       % page height aligning
%\hoffset=0cm
\pagestyle{empty}  % withour numeration

%\renewcommand{\baselinestretch}{0.5} % lowering lines interval

\begin{document}

\section*{Вопросы к экзамену по курсу архитектуры процессоров}

\begin{enumerate}
  %\setlength{\itemsep}{-1.5mm} % отступ в перечислении --- выглядит страшно

  %TODO: Use \Roman.
  \item Алгоритмы деления.
        Алгоритм Ньютона-Рафсона, вывод формулы.
  \item Алгоритмы деления.
        Алгоритм с вычитанием, деление в столбик.
  \item Сопроцессор VFPU для Allegrex.
        Особенности регистрового файла, инструкции префикса.
  \item Правила спаривания инструкций.
        Архитектура Пентиум I.
  \item Кэш данных.
        Ассоциативность кэша.
        Архитектура Пентиум I.
  \item Механизм предсказания ветвлений.
        Архитектура Пентиум I.
  \item Оптимизация кеша данных.
        Архитектура Пентиум I.
  \item Стадии конвейера.
        Архитектура Пентиум II.
  \item Механизм предсказания ветвлений.
        Архитектура Пентиум II.
  \item Особенности выполнения в не исходном порядке.
        Архитектура Пентиум II.
  \item Оптимизации декодера.
        Архитектура Пентиум II.
  \item Регистры процессора Пентиум I.
  \item Режимы адресации.
        Архитектура Пентиум I.
  \item Представление данных в процессоре с плавающей точкой.
        Архитектура Пентиум I.
  \item Регистровая организация процессора с плавающей точкой.
        Архитектура Пентиум I.
  \item Команды умножения и деления и подготовки делимого.
        Архитектура Пентиум I.
  \item Логические и сдвиговые команды.
        Архитектура Пентиум I.
  \item Команды обработки цепочек.
        Архитектура Пентиум I.
  \item Типы арифметики~--- с переполнением и насыщением.
        Примеры.
  \item Инструкции умножения, умножения и сложения.
        Архитектура ММХ.
  \item Инструкции упаковки, распаковки.
        Архитектура ММХ.
  \item Инструкции пересылки данных.
        Архитектура SSE (Пентиум III).
  \item Инструкция перемешивания.
        Архитектура SSE (Пентиум III).
  \item Инструкции управления кешем.
        Архитектура SSE (Пентиум III).
  \item Новые типы данных и примеры инструкций с ними.
        Архитектура SSE2 (Пентиум IV).
  \item Типы данных для процессора AMD K6-3Dnow! и примеры инструкций с ними.
  \item Вычисление $\frac{1}{b}$.
        Архитектура AMD K6-3Dnow!.
  \item Архитектура ARM.
        Типы арифметики, особенности инструкций.

\end{enumerate}

\end{document}
