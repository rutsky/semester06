\documentclass[a4paper,10pt]{article}

% Encoding support.
\usepackage{ucs}
\usepackage[utf8x]{inputenc}
\usepackage[T2A]{fontenc}
\usepackage[russian]{babel}

% Spaces after commas.
\frenchspacing
% Minimal carrying number of characters,
\righthyphenmin=2

\usepackage{amsmath, amsthm, amssymb}

% From K.V.Voroncov Latex in samples, 2005.
\textheight=24cm   % text height
\textwidth=16cm    % text width.
\oddsidemargin=0pt % left side indention
\topmargin=-1.5cm  % top side indention.
\parindent=24pt    % paragraph indent
\parskip=0pt       % distance between paragraphs.
\tolerance=2000
%\flushbottom       % page height aligning
%\hoffset=0cm
\pagestyle{empty}  % withour numeration

%\renewcommand{\baselinestretch}{0.5} % lowering lines interval

\begin{document}

\section*{Список вопросов к экзамену по компьютерной алгебре}

\begin{enumerate}
  %\setlength{\itemsep}{-1.5mm} % отступ в перечислении --- выглядит страшно

  \item Группы, подгруппы, гомоморфизмы групп.
        Ядро и образ гомоморфизма.
  \item Мономорфизмы, эпиморфизмы и изоморфизмы групп.
        Понятие нормального делителя (нормальной подруппы).
        Факторгруппа.
  \item Характеризация мономорфизмов в терминах ядра.
        Основная теорема о гомоморфизме.
  \item Группа подстановок (симметрическая группа).
        Четные и нечетные подстановки.
        Теорема о том, что всякая группа есть подгруппа симметрической группы (для конечных групп).
  \item Левые классы смежности по подгруппе.
        Индекс подгруппы.
        Теорема об индексе ${[G:K]} = {[G:H][H:K]}$.
  \item Действие группы на множестве.
        Орбиты.
        Разбиение множества на орбиты и формула орбит.
        Стабилизатор.
  \item Действие группы на себе сопряжениями.
        Сопряженные элементы.
        Классы сопряженности.
        Формула классов.
  \item Свободная группа.
        Теорема: всякая группа есть факторгруппа свободной группы.
  \item Прямое произведение групп.
        Свойства прямого произведения групп.
  \item Коммутативные кольца.
        Гомоморфизмы колец.
        Моно- и эпи морфизмы. % TODO: How to write this correct?
        Характеризация мономорфизмов.
  \item Идеалы и факторкольца.
        Определение простого и максимального идеала.
  \item Поля и области целостности.
        Характеризация простого и максимального идеалов в терминах факторкольца.
  \item Кольцо полиномов над полем.
        Кольца главных идеалов.
        Алгоритм Евклида в кольце полиномов.
  \item Существование максимального идеала в кольце.
        Лемма Цорна.
  \item Модули и их гомоморфизмы.
        Моно, эпи и изоморфизмы модулей. % TODO: How to write this correct?
        Примеры.
  \item Китайская теорема об остатках.
        Целочисленный вариант.
        Использование в модулярной арифметике.
  \item Общий вариант китайской теоремы об остатках.
        Применение ее к кольцу полиномов.
  \item Расширения полей.
        Конечные и алгебраические расширения.
        Теорема: любое конечное расширение является алгебраическим.
  \item Неприводимые полиномы над полем.
        Неразложимые элементы кольца.
        Понятие факториального кольца.
        Существование неприводимых полиномов над конечными полями.
  \item Характеристика поля.
        Простое подполе.
        Поля конечной характеристики.
        Конечные поля.
        Построение полей Галуа $F_q^n$.
  \item Алгебраическое замыкание поля.
        Поле разложения многочлена.
        Существование поля разложения.
        Поле Галуа, как поле разложения полинома $x^q-x$.
  \item Определение и свойства автоморфизма Фробениуса.
  \item Факториальные кольца.
        Задача о разложении полиномов на множители в кольце многочленов.
        Приведение к случаю свободного от квадратов.
  \item Теорема Берлекэмпа.
  \item Алгоритм Берлекэмпа для разложения полиномов над конечным полем.

\end{enumerate}

\end{document}
