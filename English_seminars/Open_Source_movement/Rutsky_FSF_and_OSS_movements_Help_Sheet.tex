% Rutsky_FSF_and_OSS_movements_Help_Sheet.tex
% Help Sheet on Free Software and Open Source movements for English seminars.
% Vladimir Rutsky <altsysrq@gmail.com>
% 22.03.2009

\documentclass[10pt,a4paper]{proc}

% Spaces after commas.
\frenchspacing
% Minimal carrying number of characters,
\righthyphenmin=2

% Encoding support.
\usepackage[utf8x]{inputenc}
\usepackage{url}
\usepackage[russian,english]{babel}

% From K.V.Voroncov Latex in samples, 2005.
\textheight=24cm   % Text height.
\textwidth=16cm    % Text width
\oddsidemargin=0pt % Left side indention.
\topmargin=-1.5cm  % Top side indention.
\parindent=24pt    % Paragraph indent.
\parskip=0pt       % Distance between paragraphs.
\tolerance=2000
\flushbottom       % Page height aligning.

% For defining headers.
\usepackage{fancyhdr}

% For transcriptions.
\usepackage{tipa}

% Numerating sections with Roman numerals.
\renewcommand \thesection{\Roman{section}}

% Smaller interval.
\renewcommand{\baselinestretch}{0.7}
% TODO
\newcommand{\bee}{\begin{enumerate}\setlength{\itemsep}{-0.7mm}}
\newcommand{\ene}{\end{enumerate}}
\newcommand{\bit}{\begin{itemize}\setlength{\itemsep}{-0.7mm}}
\newcommand{\eit}{\end{itemize}}

\begin{document}

% Header.
\pagestyle{empty}
\pagestyle{fancy}
\lhead{Rutsky~V.\,V., 3057/2}
\chead{Help Sheet}
\rhead{\today}
\cfoot{}

\begin{center}
{\LARGE \bfseries Free Software and Open Source movements}
\end{center}

% Content
\section{Free Software movement}

\bee
  \item Richard~M.~Stallman and his idea of Free Software.
  \item Free Software Foundation (1985).
  \item Free Software freedoms:
  \bee
    \item[Freedom 0:] The freedom to run the program, for any purpose.
    \item[Freedom 1:] The freedom to study how the program works, and adapt it to 
your needs.
    \item[Freedom 2:] The freedom to redistribute copies so you can help your 
neighbor.
    \item[Freedom 3:] The freedom to improve the program, and release your 
improvements (and modified versions in general) to the public, so that the 
whole community benefits.
  \end{enumerate}
  \item GNU Operation System.
  \item Free Software licenses.
\ene

\section{Open Source movement}
\bee
  \item Open Source Definition:
  \bee
    \item[1.] Free redistribution.
    \item[2.] Source code.
    \item[3.] Derived works.
    \item[4.] Integrity of the author's source code.
    \item[5.] No discrimination against persons or groups.
    \item[6.] No discrimination against fields of endeavor.
    \item[7.] Distribution of license.
    \item[8.] License must not be specific to a product.
    \item[9.] License must not restrict other software.
    \item[10.] License must be technology-neutral.
  \ene
  \item Open Source is a powerful and reliable software.
  \item Free Software and Open Source are different.
\ene

\section{Licenses}
\bee
  \item Copyleft vs. permissive licenses vs. public domain.
  \item GPL-like: GNU Lesser GPL, GNU Affero GPL, GNU Free Documentation License.
  \item BSD-like: BSD, MIT, Boost Software License, Apache License.
  \item CreativeCommons. CC-BY/SA/ND/NC.
\ene

\section{Free Software, Open Source and profit}
\bee
  \item ``Free'' is not about price!
  \item Selling program binary is mostly unprofitable if you are not Big Monopolist Corporation.
  \item Open Source busyness models:
  \bee
    \item Redistribution and support. Red Hat Linux.
    \item Double licensing. Trolltech~Qt, Berkeley~DB.
    \item Implementing program/servers solutions. Zend Corporation.
  \ene
\ene

\section{Stories of success}
\bee
  \item End user examples: operating systems, servers, supercomputers, desktop, web, science tools, 
a lot of developer tools\dots
  \item Corporation examples: Cygnus Solutions, Canonical~Ltd. Red~Hat, Mozilla Corporation, Qt~Software, 
Sun~Microsystems\dots
\ene

\section{Conclusion}
\bee
  \item Free Software is a way to make world better and may require some 
sacrifices.
  \item Open Source is a quality guarantee.
  \item It is possible to gain money and work for freedom.
\ene

\section*{Sources}
  \bit
    \item \url{http://www.fsf.org/}~--- Free Software Foundation
    \item \url{http://www.opensource.org/}~--- Open Source Initiative
    \item \url{http://en.wikipedia.org/wiki/Portal:Free_software}~--- Wikipedia Free Software Portal
  \eit
  
\section*{Vocabulary}
  \begin{tabular}{rll}
    proprietary & \textipa{[pr@'prai@t(@)ri]} & \Russianсобственнический \\
    license     & \textipa{['lais(@)ns]}      & \Russianлицензия \\
  \end{tabular}

\end{document}
