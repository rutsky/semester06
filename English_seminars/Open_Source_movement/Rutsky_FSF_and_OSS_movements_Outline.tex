% Rutsky_FSF_and_OSS_movements_Outline.tex
% Help Sheet on Free Software and Open Source movements for English seminars.
% Vladimir Rutsky <altsysrq@gmail.com>
% 25.03.2009

\documentclass[10pt,a4paper]{article}

% Spaces after commas.
\frenchspacing
% Minimal carrying number of characters,
\righthyphenmin=2

% Encoding support.
\usepackage[utf8x]{inputenc}
\usepackage{url}
\usepackage[russian,english]{babel}

% From K.V.Voroncov Latex in samples, 2005.
\textheight=24cm   % Text height.
\textwidth=16cm    % Text width
\oddsidemargin=0pt % Left side indention.
\topmargin=-1.5cm  % Top side indention.
\parindent=24pt    % Paragraph indent.
\parskip=0pt       % Distance between paragraphs.
\tolerance=2000
\flushbottom       % Page height aligning.


% Numerating sections with Roman numerals.
\renewcommand \thesection{\Roman{section}}

% Smaller interval.
%\renewcommand{\baselinestretch}{0.7}
% TODO
%\newcommand{\bee}{\begin{enumerate}\setlength{\itemsep}{-0.7mm}}
\newcommand{\bee}{\begin{enumerate}}
\newcommand{\ene}{\end{enumerate}}
%\newcommand{\bit}{\begin{itemize}\setlength{\itemsep}{-0.7mm}}
\newcommand{\bit}{\begin{itemize}}
\newcommand{\eit}{\end{itemize}}

\title{Free Software and Open Source movements \\ Outline}
\author{Rutsky\,V., 3057/2}
%\date{}

\begin{document}

% Title.
\maketitle

% Content.

\section{Introduction}
\bee
  \item Intellectual property (IP): copyrights, trademarks, patents, licenses.
  \bee
    \item Intellectual property rights (IPR)~--- right of owning of non-rival goods.
    \item IPR gives temporary monopoly rights.
    \item Objective of IPR: growth of technologies by stimulating process of giving to the World intellectual creations.
    \item Different types of intellectual property: copyrights, trademarks, patents, licenses.
    \item IPR and software: good idea, bad implementation.
  \ene
  \item Software categories.
  \bee
    \item Diagram showing relations between software categories.
  \ene
\ene

\section{Free Software movement}
\bee
  \item How the movement was started.
  \bee
    \item Who is Richard M. Stallman? What tendency of software development was in 1970-1980?
    \item Stallman's story about how did he went to idea of such software as Free Software.
  \ene
  \item Free Software Foundation.
  \bee
    \item Idea of Free Software and Free Software Definition, four freedoms of Free Software:
    \bee
      \item[Freedom 0:] The freedom to run the program, for any purpose.
      \item[Freedom 1:] The freedom to study how the program works, and adapt it to your needs.
      \item[Freedom 2:] The freedom to redistribute copies so you can help your neighbor.
      \item[Freedom 3:] The freedom to improve the program, and release your improvements 
      (and modified versions in general) to the public, so that the whole community benefits.
    \ene
    \item Social aspect of Free Software and the goal: it is a philosophy to make World better.
    Social solidarity: sharing and cooperation.
  \ene
  \item GNU OS.
  \bee
    \item GNU. GNU's Not Unix.
    \item Linux project and GNU/Linux OS.
  \ene
  \item Free software protectors~--- licenses.
  \bee
    \item Copyleft.
    \item GPL license.
  \ene
\ene

\section{Open Source movement}
\bee
  \item Open Source Definition:
  \bee
    % TODO: do normal counter.
    \item[1.] Free redistribution.
    \item[2.] Source code.
    \item[3.] Derived works.
    \item[4.] Integrity of the author's source code.
    \item[5.] No discrimination against persons or groups.
    \item[6.] No discrimination against fields of endeavor.
    \item[7.] Distribution of license.
    \item[8.] License must not be specific to a product.
    \item[9.] License must not restrict other software.
    \item[10.] License must be technology-neutral.
  \ene
  \item Open Source is a powerful and reliable software.
  \item Free Software and Open Source are different.
\ene

\section{Licenses}
\bee
  \item Copyleft vs. permissive licenses vs. public domain.
  \item GPL-like: GNU Lesser GPL, GNU Affero GPL, GNU Free Documentation License.
  \item BSD-like: BSD, MIT, Boost Software License, Apache License.
  \item CreativeCommons. CC-BY/SA/ND/NC.
\ene

\section{Free Software, Open Source and profit}
\bee
  \item ``Free'' is not about price!
  \item Selling program binary is mostly unprofitable if you are not Big Monopolist Corporation.
  \item Open Source business models:
  \bee
    \item Redistribution and support. Red Hat Linux.
    \item Double licensing. Trolltech~Qt, Berkeley~DB.
    \item Implementing programs/servers solutions. Zend Corporation.
  \ene
\ene

\section{Success stories}
\bee
  \item End user examples: operating systems, servers, supercomputers, desktop, web, science tools, 
a lot of developer tools\dots
  \item Corporation examples: Cygnus Solutions, Canonical~Ltd. Red~Hat, Mozilla Corporation, Qt~Software, 
Sun~Microsystems\dots
\ene

\section{Conclusion}
\bee
  \item Free Software is a way to make world better and may require some 
sacrifices.
  \item Open Source is a quality guarantee.
  \item It is possible to gain money and work for freedom.
\ene

\section*{Sources}
  \bit
    \item \url{http://www.fsf.org/}~--- Free Software Foundation
    \item \url{http://www.opensource.org/}~--- Open Source Initiative
    \item \url{http://en.wikipedia.org/wiki/Portal:Free_software}~--- Wikipedia Free Software Portal
  \eit

\end{document}
