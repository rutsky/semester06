% Rutsky_FSF_and_OSS_movements_help_sheet.tex
% Help sheet on Free Software and Open Source movements for English seminars.
% Vladimir Rutsky <altsysrq@gmail.com>
% 22.03.2009

\documentclass[10pt,a4paper]{article}

% Spaces after commas.
\frenchspacing
% Minimal carrying number of characters,
\righthyphenmin=2

% Encoding support.
\usepackage[utf8x]{inputenc}
\usepackage{url}
% Using English here ;)
%\usepackage[russian]{babel}

% From K.V.Voroncov Latex in samples, 2005.
\textheight=24cm   % Text height.
\textwidth=16cm    % Text width
\oddsidemargin=0pt % Left side indention.
\topmargin=-1.5cm  % Top side indention.
\parindent=24pt    % Paragraph indent.
\parskip=0pt       % Distance between paragraphs.
\tolerance=2000
\flushbottom       % Page height aligning.

% For defining headers.
\usepackage{fancyhdr}

% Numerating sections with Roman numerals.
\renewcommand \thesection{\Roman{section}}

\begin{document}

% Header.
\pagestyle{empty}
\pagestyle{fancy}
\lhead{V.\,V.~Rutsky, 3057/2}
\rhead{\today}

% Title
%\title{Free Software and Open Source movements}
%\maketitle

\begin{center}
{\LARGE \bfseries Free Software and Open Source movements}
\end{center}

% Content
\section{Free Software movement}

\begin{enumerate}
  \item Richard~M.~Stallman and his idea of Free Software.
  \item Free Software Foundation (1985).
  \item Free Software freedoms:
  \begin{enumerate}
    \item[Freedom 0:] The freedom to run the program, for any purpose.
    \item[Freedom 1:] The freedom to study how the program works, and adapt it to 
your needs.
    \item[Freedom 2:] The freedom to redistribute copies so you can help your 
neighbor.
    \item[Freedom 3:] The freedom to improve the program, and release your 
improvements (and modified versions in general) to the public, so that the 
whole community benefits.
  \end{enumerate}
  \item GNU Operation System.
  \item Free Software licenses.
\end{enumerate}

\section{Open Source movement}
\begin{enumerate}
  \item Open Source Definition:
  \begin{enumerate}
    \item[1.] Free redistribution.
    \item[2.] Source code.
    \item[3.] Derived works.
    \item[4.] Integrity of the author's source code.
    \item[5.] No discrimination against persons or groups.
    \item[6.] No discrimination against fields of endeavor.
    \item[7.] Distribution of license.
    \item[8.] License must not be specific to a product.
    \item[9.] License must not restrict other software.
    \item[10.] License must be technology-neutral.
  \end{enumerate}
  \item Open Source is a powerful and reliable software.
  \item Difference between Free Software and Open Source.
  \item Free Software is better than Open Source.
\end{enumerate}

\section{Licenses}
\begin{enumerate}
  \item Copyleft vs. permissive licenses vs. public domain.
  \item GPL-like: GNU Lesser GPL, GNU Affero GPL, GNU Free Documentation License.
  \item BSD-like: BSD, MIT, Boost Software License, Apache License.
  \item CreativeCommons. CC-BY/SA/ND/NC.
\end{enumerate}

\section{Free Software, Open Source and profit}
\begin{enumerate}
  \item "Free" is not about price!
  \item Selling program binary is mostly unprofitable if you are not Big Monopolist Corporation.
  \item Open Source busyness models:
  \begin{enumerate}
    \item Redistribution and support. Red Hat Linux.
    \item Double licensing. Trolltech Qt, Berkeley DB.
    \item Implementing program/servers solutions. Zend Corporation.
  \end{enumerate}
\end{enumerate}

\section{Stories of success}
\begin{enumerate}
  \item End user examples.
  \begin{enumerate}
    \item Operating systems: *BSD, GNU/Linux, ReactOS...
    \item Servers: LAMP (Linux, Apache, MySQL, PHP)...
    \item Desktop: FireFox, OpenOffice...
    \item Web: Wikimedia...
    \item Science tools: BLAS, Octave, GNU R...
    \item Languages compilers/interpreters: GCC, Python...
    \item A lot of developer tools: version control systems, build systems, tracking systems.
  \end{enumerate}
  \item Corporation examples.
  \begin{enumerate}
    \item Cygnus Solutions. "Cygnus, Your GNU Support".
    \item Canonical Ltd.
    \item Red Hat.
    \item Mozilla Corporation.
    \item Qt Software.
    \item Sun Microsystems.
  \end{enumerate}
\end{enumerate}

\section{Conclusion}
\begin{enumerate}
  \item Free Software is a way to make world better and may require some 
sacrifices.
  \item Open Source is a quality guarantee.
  \item It is possible to gain money and work for freedom.
\end{enumerate}

\end{document}
