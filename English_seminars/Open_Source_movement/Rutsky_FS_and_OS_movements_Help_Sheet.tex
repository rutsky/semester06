% Rutsky_FSF_and_OSS_movements_Help_Sheet.tex
% Help Sheet on Free Software and Open Source movements for English seminars.
% Vladimir Rutsky <altsysrq@gmail.com>
% 22.03.2009

\documentclass[10pt,a4paper]{article}

% Spaces after commas.
\frenchspacing
% Minimal carrying number of characters,
\righthyphenmin=2

% Encoding support.
\usepackage[utf8x]{inputenc}
\usepackage{url}
\usepackage[russian,english]{babel}

% From K.V.Voroncov Latex in samples, 2005.
\textheight=25cm   % Text height.
\textwidth=16cm    % Text width
\oddsidemargin=0pt % Left side indention.
\topmargin=-1.5cm  % Top side indention.
\parindent=24pt    % Paragraph indent.
\parskip=0pt       % Distance between paragraphs.
\tolerance=2000
\flushbottom       % Page height aligning.

% For defining headers.
\usepackage{fancyhdr}

% For transcriptions.
\usepackage{tipa}

% For multicolumns support.
\usepackage{multicol}

% Numerating sections with Roman numerals.
\renewcommand \thesection{\Roman{section}}

% Smaller interval.
\renewcommand{\baselinestretch}{0.60}
% TODO
\newcommand{\bee}{\begin{enumerate}\setlength{\itemsep}{-0.65mm}}
\newcommand{\ene}{\end{enumerate}}
\newcommand{\bit}{\begin{itemize}\setlength{\itemsep}{-0.65mm}}
\newcommand{\eit}{\end{itemize}}

%\title{Free Software and Open Source movements \\ Help Sheet}
%\title{Free Software and Open Source movements}
%\author{Rutsky\,V., 3057/2}
%\date{}

\begin{document}

% Title.
%\maketitle
%\thispagestyle{empty}

% Header.
\pagestyle{empty}
\pagestyle{fancy}
\rhead{Rutsky\,V., 3057/2 \qquad \today}
%\chead{Help Sheet}
\cfoot{}

\begin{center}
{\LARGE \bfseries Free Software and Open Source movements}
\end{center}

% Content.
\section*{Sources}
  \bit
    \item \url{http://www.fsf.org/}~--- Free Software Foundation
    \item \url{http://www.opensource.org/}~--- Open Source Initiative
    \item \url{http://en.wikipedia.org/wiki/Portal:Free_software}~--- Wikipedia Free Software Portal
  \eit

\begin{center}
{\Large \bfseries Contents}
\end{center}

\begin{multicols}{2}

\section*{Introduction}
\bee
  \item Intellectual property rights
  \item Software categories
\ene

\section{Free Software movement}

\bee
  \item R.M.~Stallman and his idea of Free Software
  \item Free Software freedoms:
  \bee
    \item[0.] to run the program, for any purpose;
    \item[1.] to study how the program works, and modify it;
    \item[2.] to redistribute copies;
    \item[3.] to redistribute modified copies
  \ene
  \item GNU project, GNU/Linux and Free Software Foundation
  \item Free Software licenses
\ene

\section{Open Source movement}
\bee
  \item Open Source~--- approach to software development.
  \item Open Source Definition:
  \bee
    % TODO: do normal counter.
    \item[1.] free redistribution;
    \item[2.] source code;
    \item[3.] derived works;
    \item[4.] integrity of the author's source code;
    \item[5.] no discrimination against persons or groups;
    \item[6.] no discrimination against fields of endeavor;
    \item[7.] distribution of license;
    \item[8.] license must not be specific to a product;
    \item[9.] license must not restrict other software;
    \item[10.] license must be technology-neutral
  \ene
  \item Open Source is a powerful and reliable software
  \item Free Software and Open Source are different!
\ene

\section{Licenses}
\bee
  \item Copyleft vs. permissive licenses vs. public domain
  \item GPL-like
  \item BSD-like
  \item CreativeCommons
\ene

\section{Free Software, Open Source and profit}
\bee
  \item ``Free'' is not about price!
  \item Selling program binary is mostly unprofitable if you are not Big Monopolist Corporation
  \item Open Source business models
\ene

\section{Stories of success}
\bee
  \item End user examples
  \item Corporation examples
\ene

\section*{Conclusion}
\bee
  \item Free Software is a way to make world better and may require some sacrifices
  \item Open Source is a quality guarantee and a proved model of software development
  \item It is possible to gain money and work for our freedom!
\ene

\end{multicols}

\begin{center}
{\Large \bfseries Vocabulary}
\end{center}

\begin{tabular}{lll}
  \textbf{attribution}           & \textipa{[,\ae{}tri'bju:S@n]}        & \Russian установление авторства \\
  \textbf{collaboration}         & \textipa{[k@,l\ae{}b@'reiS@n]}       & \Russian совместная работа \\
  \textbf{copyright}             & \textipa{['kOpirait]}                & \Russian авторское право \\
  \textbf{derivative}            & \textipa{[di'riv@tiv]}               & \Russian производная \\
  \textbf{ethical}               & \textipa{['eTik@l]}                  & \Russian этичный \\
  \textbf{field of endeavor}     & \textipa{[fi:ld Ov in'dev@]}         & \Russian область применения \\
  \textbf{intellectual property} & \textipa{[,inti'lektju@l 'prOp@ti]}  & \Russian интеллектуальная собственность \\
  \textbf{legal right}           & \textipa{['li:g@l rait]}             & \Russian юридическое право \\
  \textbf{patent}                & \textipa{['peit@nt]}                 & \Russian патент \\
  \textbf{permissive license}    & \textipa{[p@'misiv 'lais(@)ns]}      & \Russian либеральная лицензия \\
  \textbf{proprietary software}  & \textipa{[pr@'prai@t(@)ri 'sOftwE@]} & \Russian проприетарное (частное) программное обеспечение \\
  \textbf{public domain}         & \textipa{['pAblik d@'mein]}          & \Russian общественное достояние \\
  \textbf{reliability}           & \textipa{[ri,lai@'biliti]}           & \Russian надёжность \\
  \textbf{trade mark}            & \textipa{[treid ma:k]}               & \Russian товарный знак \\
\end{tabular}

\end{document}
