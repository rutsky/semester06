\documentclass[a4paper,10pt]{article}

% Encoding support.
\usepackage{ucs}
\usepackage[utf8x]{inputenc}
\usepackage[T2A]{fontenc}
\usepackage[russian]{babel}

% Spaces after commas.
\frenchspacing
% Minimal carrying number of characters,
\righthyphenmin=2

\usepackage{amsmath, amsthm, amssymb}

% From K.V.Voroncov Latex in samples, 2005.
\textheight=24cm   % text height
\textwidth=16cm    % text width.
\oddsidemargin=0pt % left side indention
\topmargin=-1.5cm  % top side indention.
\parindent=24pt    % paragraph indent
\parskip=0pt       % distance between paragraphs.
\tolerance=2000
%\flushbottom       % page height aligning
%\hoffset=0cm
\pagestyle{empty}  % without numeration

%\renewcommand{\baselinestretch}{0.5} % lowering lines interval

\begin{document}

\section*{Вопросы по курсу ``Математическое программирование''}

\begin{enumerate}
  %\setlength{\itemsep}{-1.5mm} % отступ в перечислении --- выглядит страшно

  \item Симплекс-метод решения задачи линейного пpoграммирования.
        Построение начального опорного вектора.
  \item Метод деления отрезка пополам.
  \item Метод золотого сечения.
  \item Метод циклического покоординатного спуска.
        Лемма о $f(x) - f(y) \geqslant f'^{\mathrm{T}}(y)(x - y) - \Lambda \|x - y \|^2$ и теорема сходимости.
  \item Грaдиентныe методы.
        Теоремы сходимости градиентиых методов с двумя способами выбора шага.
  \item Теорема o скорости сходимости градиентных методов.
  \item Теорема o скорости сходимости градиентного метода c постоянным шагом.
  \item Методы Ньютона.
        Теоремы сходимости методов Ньютона c двумя способами выбора шага.
  \item Теорема o скорости сходимости методов Ньютона.
  \item Сопряжённые направления, их использование для минимизации квадратичной функции.
  \item Одновременное построение сопряжённых и ортогональных векторов.
  \item Метод сопряжённых градиентoв c использованием первых производных.
  \item Метод сопряжённых градиентов без вычисления производных.
  \item Метод проекции градиента.
        Свойство матрицы проектирования.
        Теорема об оптимальности точки.
  \item Метод возможных направлений.
        Схема метода и обоснование выбора шага.
  \item Теорема o существовании $\eta_\delta < 0$.
  \item Алгоритм метода возможных направлений и теорема сходимости.
  \item Построение начального вектора в методе возможных направлений.
  \item Градиентный метод в задаче c ограничениями.
        Обоснование выбора шага и описание алгоритма.
  \item Теоремы сходимости и скорости сходимости метода условного гpaдиента.
  \item Метод Ньютона в задаче c ограничениями.
        Обоснование выбора шага и описание алгоритма.
  \item Теоремы сходимости и скорости сходимости метода Ньютона.
  \item Метод отсекающей гиперплоскости.
        Теорема схoдимости метода.
  \item Метод штрафных функций.
        Лемма об оценках и теорема сходимости.
  \item Метод барьерных функций.
        Лемма об оценках и теорема сходимости.

\end{enumerate}

\end{document}
