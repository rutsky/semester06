% golden_section_search.tex
% Report on using golden section search algorithm for finding one dimensional special type function minimum.
% Vladimir Rutsky <altsysrq@gmail.com>
% 24.03.2009

\documentclass[10pt,a4paper,titlepage]{article}

% Spaces after commas.
\frenchspacing
% Minimal carrying number of characters,
\righthyphenmin=2

% Encoding support.
\usepackage{ucs}
\usepackage[utf8x]{inputenc}
\usepackage[T2A]{fontenc}
\usepackage[russian]{babel}

\usepackage{amsmath, amsthm, amssymb}

% From K.V.Voroncov Latex in samples, 2005.
\textheight=24cm   % Text height.
\textwidth=16cm    % Text width
\oddsidemargin=0pt % Left side indention.
\topmargin=-1.5cm  % Top side indention.
\parindent=24pt    % Paragraph indent.
\parskip=0pt       % Distance between paragraphs.
\tolerance=2000
%\flushbottom       % Page height aligning.

% Algorithms stuff.
\usepackage{algorithm}
\usepackage{algorithmic}

\newcommand{\INPUT}{\textbf{Вход:~}}
\newcommand{\OUTPUT}{\textbf{Выход:~}}
\floatname{algorithm}{Алгоритм}
\renewcommand{\algorithmicrepeat}{\textbf{do}}
\renewcommand{\algorithmicuntil}{\textbf{while}}

% Drawing figures.
\usepackage{subfig}

% Drawing graphs.
\usepackage{tikz}
\usetikzlibrary{trees,positioning,arrows}

% For listings.
\usepackage{listings}

% Transpose symbol in math mode.
\newcommand\T{^\mathrm{\textsf{\tiny{T}}}}

% Source code listings.
\renewcommand{\lstlistingname}{Исходный код}


\title{Решение задачи одномерной минимизации функции}
\author{Владимир Руцкий, 3057/2}
\date{}

\begin{document}

\maketitle
\thispagestyle{empty}

% Content.
\section{Постановка задачи}
Требуется найти с наперёд заданной точностью минимум (локальный) одномерной функции $f(x)$ на заданном отрезке $[a, b]$:
$$ \min f(x), \quad x \in [a, b], $$
используя \textit{метод золотого сечения}.

Исходная функция: $f(x) = \frac{\cos x}{x^2}
,$ на отрезке $[7, 11].$

\section{Исследование применимости метода}
Алгоритмы поиска минимума одномерной функции $f(x)$ на отрезке $[a, b]$, 
использующие только значения функции в некоторых точках 
(исследуемый метод золотого сечения в их числе), применимы, только
если заданная функция является \textit{унимодальной}, а именно
$$ \exists ! \, x^* \in [a, b]: \quad
\begin{cases}
  \forall x_1, x_2 \in [a,b]: \quad x^* \leqslant x_1 \leqslant x_2 & \Rightarrow f(x^*) \leqslant f(x_1) \leqslant f(x_2) \\
  \forall x_1, x_2 \in [a,b]: \quad x^* \geqslant x_1 \geqslant x_2 & \Rightarrow f(x^*) \geqslant f(x_1) \geqslant f(x_2) \\
\end{cases}.
$$

\begin{figure}[h]
  \label{function-graph}
  \caption{График функции $f(x)$}
  \begin{center}
  % GNUPLOT: LaTeX picture with Postscript
\begingroup%
\makeatletter%
\newcommand{\GNUPLOTspecial}{%
  \@sanitize\catcode`\%=14\relax\special}%
\setlength{\unitlength}{0.0500bp}%
\begin{picture}(7200,5040)(0,0)%
  {\GNUPLOTspecial{"
%!PS-Adobe-2.0 EPSF-2.0
%%Title: ../output/graphic.tex
%%Creator: gnuplot 4.2 patchlevel 5 
%%CreationDate: Mon Apr 20 23:19:04 2009
%%DocumentFonts: 
%%BoundingBox: 0 0 360 252
%%EndComments
%%BeginProlog
/gnudict 256 dict def
gnudict begin
%
% The following 6 true/false flags may be edited by hand if required
% The unit line width may also be changed
%
/Color false def
/Blacktext true def
/Solid false def
/Dashlength 1 def
/Landscape false def
/Level1 false def
/Rounded false def
/TransparentPatterns false def
/gnulinewidth 5.000 def
/userlinewidth gnulinewidth def
%
/vshift -66 def
/dl1 {
  10.0 Dashlength mul mul
  Rounded { currentlinewidth 0.75 mul sub dup 0 le { pop 0.01 } if } if
} def
/dl2 {
  10.0 Dashlength mul mul
  Rounded { currentlinewidth 0.75 mul add } if
} def
/hpt_ 31.5 def
/vpt_ 31.5 def
/hpt hpt_ def
/vpt vpt_ def
Level1 {} {
/SDict 10 dict def
systemdict /pdfmark known not {
  userdict /pdfmark systemdict /cleartomark get put
} if
SDict begin [
  /Title (../output/graphic.tex)
  /Subject (gnuplot plot)
  /Creator (gnuplot 4.2 patchlevel 5 )
  /Author (Bob Rutsky,,,)
%  /Producer (gnuplot)
%  /Keywords ()
  /CreationDate (Mon Apr 20 23:19:04 2009)
  /DOCINFO pdfmark
end
} ifelse
%
% Gnuplot Prolog Version 4.2 (August 2006)
%
/M {moveto} bind def
/L {lineto} bind def
/R {rmoveto} bind def
/V {rlineto} bind def
/N {newpath moveto} bind def
/Z {closepath} bind def
/C {setrgbcolor} bind def
/f {rlineto fill} bind def
/vpt2 vpt 2 mul def
/hpt2 hpt 2 mul def
/Lshow {currentpoint stroke M 0 vshift R 
	Blacktext {gsave 0 setgray show grestore} {show} ifelse} def
/Rshow {currentpoint stroke M dup stringwidth pop neg vshift R
	Blacktext {gsave 0 setgray show grestore} {show} ifelse} def
/Cshow {currentpoint stroke M dup stringwidth pop -2 div vshift R 
	Blacktext {gsave 0 setgray show grestore} {show} ifelse} def
/UP {dup vpt_ mul /vpt exch def hpt_ mul /hpt exch def
  /hpt2 hpt 2 mul def /vpt2 vpt 2 mul def} def
/DL {Color {setrgbcolor Solid {pop []} if 0 setdash}
 {pop pop pop 0 setgray Solid {pop []} if 0 setdash} ifelse} def
/BL {stroke userlinewidth 2 mul setlinewidth
	Rounded {1 setlinejoin 1 setlinecap} if} def
/AL {stroke userlinewidth 2 div setlinewidth
	Rounded {1 setlinejoin 1 setlinecap} if} def
/UL {dup gnulinewidth mul /userlinewidth exch def
	dup 1 lt {pop 1} if 10 mul /udl exch def} def
/PL {stroke userlinewidth setlinewidth
	Rounded {1 setlinejoin 1 setlinecap} if} def
% Default Line colors
/LCw {1 1 1} def
/LCb {0 0 0} def
/LCa {0 0 0} def
/LC0 {1 0 0} def
/LC1 {0 1 0} def
/LC2 {0 0 1} def
/LC3 {1 0 1} def
/LC4 {0 1 1} def
/LC5 {1 1 0} def
/LC6 {0 0 0} def
/LC7 {1 0.3 0} def
/LC8 {0.5 0.5 0.5} def
% Default Line Types
/LTw {PL [] 1 setgray} def
/LTb {BL [] LCb DL} def
/LTa {AL [1 udl mul 2 udl mul] 0 setdash LCa setrgbcolor} def
/LT0 {PL [] LC0 DL} def
/LT1 {PL [4 dl1 2 dl2] LC1 DL} def
/LT2 {PL [2 dl1 3 dl2] LC2 DL} def
/LT3 {PL [1 dl1 1.5 dl2] LC3 DL} def
/LT4 {PL [6 dl1 2 dl2 1 dl1 2 dl2] LC4 DL} def
/LT5 {PL [3 dl1 3 dl2 1 dl1 3 dl2] LC5 DL} def
/LT6 {PL [2 dl1 2 dl2 2 dl1 6 dl2] LC6 DL} def
/LT7 {PL [1 dl1 2 dl2 6 dl1 2 dl2 1 dl1 2 dl2] LC7 DL} def
/LT8 {PL [2 dl1 2 dl2 2 dl1 2 dl2 2 dl1 2 dl2 2 dl1 4 dl2] LC8 DL} def
/Pnt {stroke [] 0 setdash gsave 1 setlinecap M 0 0 V stroke grestore} def
/Dia {stroke [] 0 setdash 2 copy vpt add M
  hpt neg vpt neg V hpt vpt neg V
  hpt vpt V hpt neg vpt V closepath stroke
  Pnt} def
/Pls {stroke [] 0 setdash vpt sub M 0 vpt2 V
  currentpoint stroke M
  hpt neg vpt neg R hpt2 0 V stroke
 } def
/Box {stroke [] 0 setdash 2 copy exch hpt sub exch vpt add M
  0 vpt2 neg V hpt2 0 V 0 vpt2 V
  hpt2 neg 0 V closepath stroke
  Pnt} def
/Crs {stroke [] 0 setdash exch hpt sub exch vpt add M
  hpt2 vpt2 neg V currentpoint stroke M
  hpt2 neg 0 R hpt2 vpt2 V stroke} def
/TriU {stroke [] 0 setdash 2 copy vpt 1.12 mul add M
  hpt neg vpt -1.62 mul V
  hpt 2 mul 0 V
  hpt neg vpt 1.62 mul V closepath stroke
  Pnt} def
/Star {2 copy Pls Crs} def
/BoxF {stroke [] 0 setdash exch hpt sub exch vpt add M
  0 vpt2 neg V hpt2 0 V 0 vpt2 V
  hpt2 neg 0 V closepath fill} def
/TriUF {stroke [] 0 setdash vpt 1.12 mul add M
  hpt neg vpt -1.62 mul V
  hpt 2 mul 0 V
  hpt neg vpt 1.62 mul V closepath fill} def
/TriD {stroke [] 0 setdash 2 copy vpt 1.12 mul sub M
  hpt neg vpt 1.62 mul V
  hpt 2 mul 0 V
  hpt neg vpt -1.62 mul V closepath stroke
  Pnt} def
/TriDF {stroke [] 0 setdash vpt 1.12 mul sub M
  hpt neg vpt 1.62 mul V
  hpt 2 mul 0 V
  hpt neg vpt -1.62 mul V closepath fill} def
/DiaF {stroke [] 0 setdash vpt add M
  hpt neg vpt neg V hpt vpt neg V
  hpt vpt V hpt neg vpt V closepath fill} def
/Pent {stroke [] 0 setdash 2 copy gsave
  translate 0 hpt M 4 {72 rotate 0 hpt L} repeat
  closepath stroke grestore Pnt} def
/PentF {stroke [] 0 setdash gsave
  translate 0 hpt M 4 {72 rotate 0 hpt L} repeat
  closepath fill grestore} def
/Circle {stroke [] 0 setdash 2 copy
  hpt 0 360 arc stroke Pnt} def
/CircleF {stroke [] 0 setdash hpt 0 360 arc fill} def
/C0 {BL [] 0 setdash 2 copy moveto vpt 90 450 arc} bind def
/C1 {BL [] 0 setdash 2 copy moveto
	2 copy vpt 0 90 arc closepath fill
	vpt 0 360 arc closepath} bind def
/C2 {BL [] 0 setdash 2 copy moveto
	2 copy vpt 90 180 arc closepath fill
	vpt 0 360 arc closepath} bind def
/C3 {BL [] 0 setdash 2 copy moveto
	2 copy vpt 0 180 arc closepath fill
	vpt 0 360 arc closepath} bind def
/C4 {BL [] 0 setdash 2 copy moveto
	2 copy vpt 180 270 arc closepath fill
	vpt 0 360 arc closepath} bind def
/C5 {BL [] 0 setdash 2 copy moveto
	2 copy vpt 0 90 arc
	2 copy moveto
	2 copy vpt 180 270 arc closepath fill
	vpt 0 360 arc} bind def
/C6 {BL [] 0 setdash 2 copy moveto
	2 copy vpt 90 270 arc closepath fill
	vpt 0 360 arc closepath} bind def
/C7 {BL [] 0 setdash 2 copy moveto
	2 copy vpt 0 270 arc closepath fill
	vpt 0 360 arc closepath} bind def
/C8 {BL [] 0 setdash 2 copy moveto
	2 copy vpt 270 360 arc closepath fill
	vpt 0 360 arc closepath} bind def
/C9 {BL [] 0 setdash 2 copy moveto
	2 copy vpt 270 450 arc closepath fill
	vpt 0 360 arc closepath} bind def
/C10 {BL [] 0 setdash 2 copy 2 copy moveto vpt 270 360 arc closepath fill
	2 copy moveto
	2 copy vpt 90 180 arc closepath fill
	vpt 0 360 arc closepath} bind def
/C11 {BL [] 0 setdash 2 copy moveto
	2 copy vpt 0 180 arc closepath fill
	2 copy moveto
	2 copy vpt 270 360 arc closepath fill
	vpt 0 360 arc closepath} bind def
/C12 {BL [] 0 setdash 2 copy moveto
	2 copy vpt 180 360 arc closepath fill
	vpt 0 360 arc closepath} bind def
/C13 {BL [] 0 setdash 2 copy moveto
	2 copy vpt 0 90 arc closepath fill
	2 copy moveto
	2 copy vpt 180 360 arc closepath fill
	vpt 0 360 arc closepath} bind def
/C14 {BL [] 0 setdash 2 copy moveto
	2 copy vpt 90 360 arc closepath fill
	vpt 0 360 arc} bind def
/C15 {BL [] 0 setdash 2 copy vpt 0 360 arc closepath fill
	vpt 0 360 arc closepath} bind def
/Rec {newpath 4 2 roll moveto 1 index 0 rlineto 0 exch rlineto
	neg 0 rlineto closepath} bind def
/Square {dup Rec} bind def
/Bsquare {vpt sub exch vpt sub exch vpt2 Square} bind def
/S0 {BL [] 0 setdash 2 copy moveto 0 vpt rlineto BL Bsquare} bind def
/S1 {BL [] 0 setdash 2 copy vpt Square fill Bsquare} bind def
/S2 {BL [] 0 setdash 2 copy exch vpt sub exch vpt Square fill Bsquare} bind def
/S3 {BL [] 0 setdash 2 copy exch vpt sub exch vpt2 vpt Rec fill Bsquare} bind def
/S4 {BL [] 0 setdash 2 copy exch vpt sub exch vpt sub vpt Square fill Bsquare} bind def
/S5 {BL [] 0 setdash 2 copy 2 copy vpt Square fill
	exch vpt sub exch vpt sub vpt Square fill Bsquare} bind def
/S6 {BL [] 0 setdash 2 copy exch vpt sub exch vpt sub vpt vpt2 Rec fill Bsquare} bind def
/S7 {BL [] 0 setdash 2 copy exch vpt sub exch vpt sub vpt vpt2 Rec fill
	2 copy vpt Square fill Bsquare} bind def
/S8 {BL [] 0 setdash 2 copy vpt sub vpt Square fill Bsquare} bind def
/S9 {BL [] 0 setdash 2 copy vpt sub vpt vpt2 Rec fill Bsquare} bind def
/S10 {BL [] 0 setdash 2 copy vpt sub vpt Square fill 2 copy exch vpt sub exch vpt Square fill
	Bsquare} bind def
/S11 {BL [] 0 setdash 2 copy vpt sub vpt Square fill 2 copy exch vpt sub exch vpt2 vpt Rec fill
	Bsquare} bind def
/S12 {BL [] 0 setdash 2 copy exch vpt sub exch vpt sub vpt2 vpt Rec fill Bsquare} bind def
/S13 {BL [] 0 setdash 2 copy exch vpt sub exch vpt sub vpt2 vpt Rec fill
	2 copy vpt Square fill Bsquare} bind def
/S14 {BL [] 0 setdash 2 copy exch vpt sub exch vpt sub vpt2 vpt Rec fill
	2 copy exch vpt sub exch vpt Square fill Bsquare} bind def
/S15 {BL [] 0 setdash 2 copy Bsquare fill Bsquare} bind def
/D0 {gsave translate 45 rotate 0 0 S0 stroke grestore} bind def
/D1 {gsave translate 45 rotate 0 0 S1 stroke grestore} bind def
/D2 {gsave translate 45 rotate 0 0 S2 stroke grestore} bind def
/D3 {gsave translate 45 rotate 0 0 S3 stroke grestore} bind def
/D4 {gsave translate 45 rotate 0 0 S4 stroke grestore} bind def
/D5 {gsave translate 45 rotate 0 0 S5 stroke grestore} bind def
/D6 {gsave translate 45 rotate 0 0 S6 stroke grestore} bind def
/D7 {gsave translate 45 rotate 0 0 S7 stroke grestore} bind def
/D8 {gsave translate 45 rotate 0 0 S8 stroke grestore} bind def
/D9 {gsave translate 45 rotate 0 0 S9 stroke grestore} bind def
/D10 {gsave translate 45 rotate 0 0 S10 stroke grestore} bind def
/D11 {gsave translate 45 rotate 0 0 S11 stroke grestore} bind def
/D12 {gsave translate 45 rotate 0 0 S12 stroke grestore} bind def
/D13 {gsave translate 45 rotate 0 0 S13 stroke grestore} bind def
/D14 {gsave translate 45 rotate 0 0 S14 stroke grestore} bind def
/D15 {gsave translate 45 rotate 0 0 S15 stroke grestore} bind def
/DiaE {stroke [] 0 setdash vpt add M
  hpt neg vpt neg V hpt vpt neg V
  hpt vpt V hpt neg vpt V closepath stroke} def
/BoxE {stroke [] 0 setdash exch hpt sub exch vpt add M
  0 vpt2 neg V hpt2 0 V 0 vpt2 V
  hpt2 neg 0 V closepath stroke} def
/TriUE {stroke [] 0 setdash vpt 1.12 mul add M
  hpt neg vpt -1.62 mul V
  hpt 2 mul 0 V
  hpt neg vpt 1.62 mul V closepath stroke} def
/TriDE {stroke [] 0 setdash vpt 1.12 mul sub M
  hpt neg vpt 1.62 mul V
  hpt 2 mul 0 V
  hpt neg vpt -1.62 mul V closepath stroke} def
/PentE {stroke [] 0 setdash gsave
  translate 0 hpt M 4 {72 rotate 0 hpt L} repeat
  closepath stroke grestore} def
/CircE {stroke [] 0 setdash 
  hpt 0 360 arc stroke} def
/Opaque {gsave closepath 1 setgray fill grestore 0 setgray closepath} def
/DiaW {stroke [] 0 setdash vpt add M
  hpt neg vpt neg V hpt vpt neg V
  hpt vpt V hpt neg vpt V Opaque stroke} def
/BoxW {stroke [] 0 setdash exch hpt sub exch vpt add M
  0 vpt2 neg V hpt2 0 V 0 vpt2 V
  hpt2 neg 0 V Opaque stroke} def
/TriUW {stroke [] 0 setdash vpt 1.12 mul add M
  hpt neg vpt -1.62 mul V
  hpt 2 mul 0 V
  hpt neg vpt 1.62 mul V Opaque stroke} def
/TriDW {stroke [] 0 setdash vpt 1.12 mul sub M
  hpt neg vpt 1.62 mul V
  hpt 2 mul 0 V
  hpt neg vpt -1.62 mul V Opaque stroke} def
/PentW {stroke [] 0 setdash gsave
  translate 0 hpt M 4 {72 rotate 0 hpt L} repeat
  Opaque stroke grestore} def
/CircW {stroke [] 0 setdash 
  hpt 0 360 arc Opaque stroke} def
/BoxFill {gsave Rec 1 setgray fill grestore} def
/Density {
  /Fillden exch def
  currentrgbcolor
  /ColB exch def /ColG exch def /ColR exch def
  /ColR ColR Fillden mul Fillden sub 1 add def
  /ColG ColG Fillden mul Fillden sub 1 add def
  /ColB ColB Fillden mul Fillden sub 1 add def
  ColR ColG ColB setrgbcolor} def
/BoxColFill {gsave Rec PolyFill} def
/PolyFill {gsave Density fill grestore grestore} def
/h {rlineto rlineto rlineto gsave closepath fill grestore} bind def
%
% PostScript Level 1 Pattern Fill routine for rectangles
% Usage: x y w h s a XX PatternFill
%	x,y = lower left corner of box to be filled
%	w,h = width and height of box
%	  a = angle in degrees between lines and x-axis
%	 XX = 0/1 for no/yes cross-hatch
%
/PatternFill {gsave /PFa [ 9 2 roll ] def
  PFa 0 get PFa 2 get 2 div add PFa 1 get PFa 3 get 2 div add translate
  PFa 2 get -2 div PFa 3 get -2 div PFa 2 get PFa 3 get Rec
  gsave 1 setgray fill grestore clip
  currentlinewidth 0.5 mul setlinewidth
  /PFs PFa 2 get dup mul PFa 3 get dup mul add sqrt def
  0 0 M PFa 5 get rotate PFs -2 div dup translate
  0 1 PFs PFa 4 get div 1 add floor cvi
	{PFa 4 get mul 0 M 0 PFs V} for
  0 PFa 6 get ne {
	0 1 PFs PFa 4 get div 1 add floor cvi
	{PFa 4 get mul 0 2 1 roll M PFs 0 V} for
 } if
  stroke grestore} def
%
/languagelevel where
 {pop languagelevel} {1} ifelse
 2 lt
	{/InterpretLevel1 true def}
	{/InterpretLevel1 Level1 def}
 ifelse
%
% PostScript level 2 pattern fill definitions
%
/Level2PatternFill {
/Tile8x8 {/PaintType 2 /PatternType 1 /TilingType 1 /BBox [0 0 8 8] /XStep 8 /YStep 8}
	bind def
/KeepColor {currentrgbcolor [/Pattern /DeviceRGB] setcolorspace} bind def
<< Tile8x8
 /PaintProc {0.5 setlinewidth pop 0 0 M 8 8 L 0 8 M 8 0 L stroke} 
>> matrix makepattern
/Pat1 exch def
<< Tile8x8
 /PaintProc {0.5 setlinewidth pop 0 0 M 8 8 L 0 8 M 8 0 L stroke
	0 4 M 4 8 L 8 4 L 4 0 L 0 4 L stroke}
>> matrix makepattern
/Pat2 exch def
<< Tile8x8
 /PaintProc {0.5 setlinewidth pop 0 0 M 0 8 L
	8 8 L 8 0 L 0 0 L fill}
>> matrix makepattern
/Pat3 exch def
<< Tile8x8
 /PaintProc {0.5 setlinewidth pop -4 8 M 8 -4 L
	0 12 M 12 0 L stroke}
>> matrix makepattern
/Pat4 exch def
<< Tile8x8
 /PaintProc {0.5 setlinewidth pop -4 0 M 8 12 L
	0 -4 M 12 8 L stroke}
>> matrix makepattern
/Pat5 exch def
<< Tile8x8
 /PaintProc {0.5 setlinewidth pop -2 8 M 4 -4 L
	0 12 M 8 -4 L 4 12 M 10 0 L stroke}
>> matrix makepattern
/Pat6 exch def
<< Tile8x8
 /PaintProc {0.5 setlinewidth pop -2 0 M 4 12 L
	0 -4 M 8 12 L 4 -4 M 10 8 L stroke}
>> matrix makepattern
/Pat7 exch def
<< Tile8x8
 /PaintProc {0.5 setlinewidth pop 8 -2 M -4 4 L
	12 0 M -4 8 L 12 4 M 0 10 L stroke}
>> matrix makepattern
/Pat8 exch def
<< Tile8x8
 /PaintProc {0.5 setlinewidth pop 0 -2 M 12 4 L
	-4 0 M 12 8 L -4 4 M 8 10 L stroke}
>> matrix makepattern
/Pat9 exch def
/Pattern1 {PatternBgnd KeepColor Pat1 setpattern} bind def
/Pattern2 {PatternBgnd KeepColor Pat2 setpattern} bind def
/Pattern3 {PatternBgnd KeepColor Pat3 setpattern} bind def
/Pattern4 {PatternBgnd KeepColor Landscape {Pat5} {Pat4} ifelse setpattern} bind def
/Pattern5 {PatternBgnd KeepColor Landscape {Pat4} {Pat5} ifelse setpattern} bind def
/Pattern6 {PatternBgnd KeepColor Landscape {Pat9} {Pat6} ifelse setpattern} bind def
/Pattern7 {PatternBgnd KeepColor Landscape {Pat8} {Pat7} ifelse setpattern} bind def
} def
%
%
%End of PostScript Level 2 code
%
/PatternBgnd {
  TransparentPatterns {} {gsave 1 setgray fill grestore} ifelse
} def
%
% Substitute for Level 2 pattern fill codes with
% grayscale if Level 2 support is not selected.
%
/Level1PatternFill {
/Pattern1 {0.250 Density} bind def
/Pattern2 {0.500 Density} bind def
/Pattern3 {0.750 Density} bind def
/Pattern4 {0.125 Density} bind def
/Pattern5 {0.375 Density} bind def
/Pattern6 {0.625 Density} bind def
/Pattern7 {0.875 Density} bind def
} def
%
% Now test for support of Level 2 code
%
Level1 {Level1PatternFill} {Level2PatternFill} ifelse
%
/Symbol-Oblique /Symbol findfont [1 0 .167 1 0 0] makefont
dup length dict begin {1 index /FID eq {pop pop} {def} ifelse} forall
currentdict end definefont pop
end
%%EndProlog
gnudict begin
gsave
0 0 translate
0.050 0.050 scale
0 setgray
newpath
gsave % colour palette begin
/maxcolors 64 def
/HSV2RGB {  exch dup 0.0 eq {pop exch pop dup dup} % achromatic gray
  { /HSVs exch def /HSVv exch def 6.0 mul dup floor dup 3 1 roll sub
     /HSVf exch def /HSVi exch cvi def /HSVp HSVv 1.0 HSVs sub mul def
	 /HSVq HSVv 1.0 HSVs HSVf mul sub mul def 
	 /HSVt HSVv 1.0 HSVs 1.0 HSVf sub mul sub mul def
	 /HSVi HSVi 6 mod def 0 HSVi eq {HSVv HSVt HSVp}
	 {1 HSVi eq {HSVq HSVv HSVp}{2 HSVi eq {HSVp HSVv HSVt}
	 {3 HSVi eq {HSVp HSVq HSVv}{4 HSVi eq {HSVt HSVp HSVv}
	 {HSVv HSVp HSVq} ifelse} ifelse} ifelse} ifelse} ifelse
  } ifelse} def
/Constrain {
  dup 0 lt {0 exch pop}{dup 1 gt {1 exch pop} if} ifelse} def
/YIQ2RGB {
  3 copy -1.702 mul exch -1.105 mul add add Constrain 4 1 roll
  3 copy -0.647 mul exch -0.272 mul add add Constrain 5 1 roll
  0.621 mul exch -0.956 mul add add Constrain 3 1 roll } def
/CMY2RGB {  1 exch sub exch 1 exch sub 3 2 roll 1 exch sub 3 1 roll exch } def
/XYZ2RGB {  3 copy -0.9017 mul exch -0.1187 mul add exch 0.0585 mul exch add
  Constrain 4 1 roll 3 copy -0.0279 mul exch 1.999 mul add exch
  -0.9844 mul add Constrain 5 1 roll -0.2891 mul exch -0.5338 mul add
  exch 1.91 mul exch add Constrain 3 1 roll} def
/SelectSpace {ColorSpace (HSV) eq {HSV2RGB}{ColorSpace (XYZ) eq {
  XYZ2RGB}{ColorSpace (CMY) eq {CMY2RGB}{ColorSpace (YIQ) eq {YIQ2RGB}
  if} ifelse} ifelse} ifelse} def
/InterpolatedColor true def
/grayindex {/gidx 0 def
  {GrayA gidx get grayv ge {exit} if /gidx gidx 1 add def} loop} def
/dgdx {grayv GrayA gidx get sub GrayA gidx 1 sub get
  GrayA gidx get sub div} def 
/redvalue {RedA gidx get RedA gidx 1 sub get
  RedA gidx get sub dgdxval mul add} def
/greenvalue {GreenA gidx get GreenA gidx 1 sub get
  GreenA gidx get sub dgdxval mul add} def
/bluevalue {BlueA gidx get BlueA gidx 1 sub get
  BlueA gidx get sub dgdxval mul add} def
/interpolate {
  grayindex grayv GrayA gidx get sub abs 1e-5 le
    {RedA gidx get GreenA gidx get BlueA gidx get}
    {/dgdxval dgdx def redvalue greenvalue bluevalue} ifelse} def
/GrayA [0 .0159 .0317 .0476 .0635 .0794 .0952 .1111 .127 .1429 .1587 .1746 
  .1905 .2063 .2222 .2381 .254 .2698 .2857 .3016 .3175 .3333 .3492 .3651 
  .381 .3968 .4127 .4286 .4444 .4603 .4762 .4921 .5079 .5238 .5397 .5556 
  .5714 .5873 .6032 .619 .6349 .6508 .6667 .6825 .6984 .7143 .7302 .746 
  .7619 .7778 .7937 .8095 .8254 .8413 .8571 .873 .8889 .9048 .9206 .9365 
  .9524 .9683 .9841 1 ] def
/RedA [0 0 0 0 0 0 0 0 0 0 0 0 0 0 0 0 0 0 0 0 0 0 0 0 .0238 .0873 .1508 
  .2143 .2778 .3413 .4048 .4683 .5317 .5952 .6587 .7222 .7857 .8492 .9127 
  .9762 1 1 1 1 1 1 1 1 1 1 1 1 1 1 1 1 .9444 .881 .8175 .754 .6905 .627 
  .5635 .5 ] def
/GreenA [0 0 0 0 0 0 0 0 .0079 .0714 .1349 .1984 .2619 .3254 .3889 .4524 
  .5159 .5794 .6429 .7063 .7698 .8333 .8968 .9603 1 1 1 1 1 1 1 1 1 1 1 1 1 
  1 1 1 .9603 .8968 .8333 .7698 .7063 .6429 .5794 .5159 .4524 .3889 .3254 
  .2619 .1984 .1349 .0714 .0079 0 0 0 0 0 0 0 0 ] def
/BlueA [.5 .5635 .627 .6905 .754 .8175 .881 .9444 1 1 1 1 1 1 1 1 1 1 1 1 1 
  1 1 1 .9762 .9127 .8492 .7857 .7222 .6587 .5952 .5317 .4683 .4048 .3413 
  .2778 .2143 .1508 .0873 .0238 0 0 0 0 0 0 0 0 0 0 0 0 0 0 0 0 0 0 0 0 0 0 
  0 0 ] def
/pm3dround {maxcolors 0 gt {dup 1 ge
	{pop 1} {maxcolors mul floor maxcolors 1 sub div} ifelse} if} def
/pm3dGamma 1.0 1.5 div def
/ColorSpace (RGB) def
Color InterpolatedColor or { % COLOUR vs. GRAY map
  InterpolatedColor { %% Interpolation vs. RGB-Formula
    /g {stroke pm3dround /grayv exch def interpolate
        SelectSpace setrgbcolor} bind def
  }{
  /g {stroke pm3dround dup cF7 Constrain exch dup cF5 Constrain exch cF15 Constrain 
       SelectSpace setrgbcolor} bind def
  } ifelse
}{
  /g {stroke pm3dround pm3dGamma exp setgray} bind def
} ifelse
0.500 UL
LTb
1260 714 M
63 0 V
5577 0 R
-63 0 V
stroke
0.00 0.00 0.00 C 0.500 UL
LTb
1260 1500 M
63 0 V
5577 0 R
-63 0 V
stroke
0.00 0.00 0.00 C 0.500 UL
LTb
1260 2286 M
63 0 V
5577 0 R
-63 0 V
stroke
0.00 0.00 0.00 C 0.500 UL
LTb
1260 3071 M
63 0 V
5577 0 R
-63 0 V
stroke
0.00 0.00 0.00 C 0.500 UL
LTb
1260 3857 M
63 0 V
5577 0 R
-63 0 V
stroke
0.00 0.00 0.00 C 0.500 UL
LTb
1260 4643 M
63 0 V
5577 0 R
-63 0 V
stroke
0.00 0.00 0.00 C 0.500 UL
LTb
1260 400 M
0 63 V
0 4337 R
0 -63 V
stroke
0.00 0.00 0.00 C 0.500 UL
LTb
1965 400 M
0 63 V
0 4337 R
0 -63 V
stroke
0.00 0.00 0.00 C 0.500 UL
LTb
2670 400 M
0 63 V
0 4337 R
0 -63 V
stroke
0.00 0.00 0.00 C 0.500 UL
LTb
3375 400 M
0 63 V
0 4337 R
0 -63 V
stroke
0.00 0.00 0.00 C 0.500 UL
LTb
4080 400 M
0 63 V
0 4337 R
0 -63 V
stroke
0.00 0.00 0.00 C 0.500 UL
LTb
4785 400 M
0 63 V
0 4337 R
0 -63 V
stroke
0.00 0.00 0.00 C 0.500 UL
LTb
5490 400 M
0 63 V
0 4337 R
0 -63 V
stroke
0.00 0.00 0.00 C 0.500 UL
LTb
6195 400 M
0 63 V
0 4337 R
0 -63 V
stroke
0.00 0.00 0.00 C 0.500 UL
LTb
6900 400 M
0 63 V
0 4337 R
0 -63 V
stroke
0.00 0.00 0.00 C 0.500 UL
LTb
0.500 UL
LTb
1260 4800 N
0 -4400 V
5640 0 V
0 4400 V
-5640 0 V
Z stroke
1.000 UP
0.500 UL
LTb
0.500 UL
LT0
0.00 0.00 0.00 C 1260 4703 M
1 -2 V
2 -3 V
1 -3 V
2 -3 V
1 -3 V
1 -2 V
2 -3 V
1 -3 V
2 -3 V
1 -3 V
2 -2 V
1 -3 V
1 -3 V
2 -3 V
1 -3 V
2 -2 V
1 -3 V
1 -3 V
2 -3 V
1 -3 V
2 -2 V
1 -3 V
1 -3 V
2 -3 V
1 -3 V
2 -3 V
1 -2 V
1 -3 V
2 -3 V
1 -3 V
2 -3 V
1 -3 V
2 -2 V
1 -3 V
1 -3 V
2 -3 V
1 -3 V
2 -2 V
1 -3 V
1 -3 V
2 -3 V
1 -3 V
2 -3 V
1 -2 V
1 -3 V
2 -3 V
1 -3 V
2 -3 V
1 -3 V
1 -2 V
2 -3 V
1 -3 V
2 -3 V
1 -3 V
2 -3 V
1 -3 V
1 -2 V
2 -3 V
1 -3 V
2 -3 V
1 -3 V
1 -3 V
2 -2 V
1 -3 V
2 -3 V
1 -3 V
1 -3 V
2 -3 V
1 -3 V
2 -2 V
1 -3 V
2 -3 V
1 -3 V
1 -3 V
2 -3 V
1 -3 V
2 -2 V
1 -3 V
1 -3 V
2 -3 V
1 -3 V
2 -3 V
1 -3 V
1 -2 V
2 -3 V
1 -3 V
2 -3 V
1 -3 V
1 -3 V
2 -3 V
1 -3 V
2 -2 V
1 -3 V
2 -3 V
1 -3 V
1 -3 V
2 -3 V
1 -3 V
2 -2 V
1 -3 V
1 -3 V
2 -3 V
1 -3 V
2 -3 V
stroke 1407 4408 M
1 -3 V
1 -3 V
2 -2 V
1 -3 V
2 -3 V
1 -3 V
2 -3 V
1 -3 V
1 -3 V
2 -3 V
1 -3 V
2 -2 V
1 -3 V
1 -3 V
2 -3 V
1 -3 V
2 -3 V
1 -3 V
1 -3 V
2 -2 V
1 -3 V
2 -3 V
1 -3 V
1 -3 V
2 -3 V
1 -3 V
2 -3 V
1 -3 V
2 -2 V
1 -3 V
1 -3 V
2 -3 V
1 -3 V
2 -3 V
1 -3 V
1 -3 V
2 -3 V
1 -3 V
2 -2 V
1 -3 V
1 -3 V
2 -3 V
1 -3 V
2 -3 V
1 -3 V
2 -3 V
1 -3 V
1 -3 V
2 -2 V
1 -3 V
2 -3 V
1 -3 V
1 -3 V
2 -3 V
1 -3 V
2 -3 V
1 -3 V
1 -3 V
2 -2 V
1 -3 V
2 -3 V
1 -3 V
1 -3 V
2 -3 V
1 -3 V
2 -3 V
1 -3 V
2 -3 V
1 -3 V
1 -2 V
2 -3 V
1 -3 V
2 -3 V
1 -3 V
1 -3 V
2 -3 V
1 -3 V
2 -3 V
1 -3 V
1 -3 V
2 -2 V
1 -3 V
2 -3 V
1 -3 V
1 -3 V
2 -3 V
1 -3 V
2 -3 V
1 -3 V
2 -3 V
1 -3 V
1 -3 V
2 -2 V
1 -3 V
2 -3 V
1 -3 V
1 -3 V
2 -3 V
1 -3 V
2 -3 V
1 -3 V
1 -3 V
2 -3 V
1 -3 V
stroke 1553 4106 M
2 -2 V
1 -3 V
2 -3 V
1 -3 V
1 -3 V
2 -3 V
1 -3 V
2 -3 V
1 -3 V
1 -3 V
2 -3 V
1 -3 V
2 -3 V
1 -2 V
1 -3 V
2 -3 V
1 -3 V
2 -3 V
1 -3 V
1 -3 V
2 -3 V
1 -3 V
2 -3 V
1 -3 V
2 -3 V
1 -3 V
1 -2 V
2 -3 V
1 -3 V
2 -3 V
1 -3 V
1 -3 V
2 -3 V
1 -3 V
2 -3 V
1 -3 V
1 -3 V
2 -3 V
1 -3 V
2 -3 V
1 -2 V
2 -3 V
1 -3 V
1 -3 V
2 -3 V
1 -3 V
2 -3 V
1 -3 V
1 -3 V
2 -3 V
1 -3 V
2 -3 V
1 -3 V
1 -3 V
2 -2 V
1 -3 V
2 -3 V
1 -3 V
1 -3 V
2 -3 V
1 -3 V
2 -3 V
1 -3 V
2 -3 V
1 -3 V
1 -3 V
2 -3 V
1 -3 V
2 -3 V
1 -2 V
1 -3 V
2 -3 V
1 -3 V
2 -3 V
1 -3 V
1 -3 V
2 -3 V
1 -3 V
2 -3 V
1 -3 V
1 -3 V
2 -3 V
1 -3 V
2 -3 V
1 -2 V
2 -3 V
1 -3 V
1 -3 V
2 -3 V
1 -3 V
2 -3 V
1 -3 V
1 -3 V
2 -3 V
1 -3 V
2 -3 V
1 -3 V
1 -3 V
2 -3 V
1 -2 V
2 -3 V
1 -3 V
2 -3 V
1 -3 V
stroke 1700 3802 M
1 -3 V
2 -3 V
1 -3 V
2 -3 V
1 -3 V
1 -3 V
2 -3 V
1 -3 V
2 -3 V
1 -2 V
1 -3 V
2 -3 V
1 -3 V
2 -3 V
1 -3 V
1 -3 V
2 -3 V
1 -3 V
2 -3 V
1 -3 V
2 -3 V
1 -3 V
1 -3 V
2 -3 V
1 -2 V
2 -3 V
1 -3 V
1 -3 V
2 -3 V
1 -3 V
2 -3 V
1 -3 V
1 -3 V
2 -3 V
1 -3 V
2 -3 V
1 -3 V
1 -3 V
2 -2 V
1 -3 V
2 -3 V
1 -3 V
2 -3 V
1 -3 V
1 -3 V
2 -3 V
1 -3 V
2 -3 V
1 -3 V
1 -3 V
2 -3 V
1 -3 V
2 -2 V
1 -3 V
1 -3 V
2 -3 V
1 -3 V
2 -3 V
1 -3 V
2 -3 V
1 -3 V
1 -3 V
2 -3 V
1 -3 V
2 -3 V
1 -3 V
1 -2 V
2 -3 V
1 -3 V
2 -3 V
1 -3 V
1 -3 V
2 -3 V
1 -3 V
2 -3 V
1 -3 V
1 -3 V
2 -3 V
1 -3 V
2 -2 V
1 -3 V
2 -3 V
1 -3 V
1 -3 V
2 -3 V
1 -3 V
2 -3 V
1 -3 V
1 -3 V
2 -3 V
1 -3 V
2 -2 V
1 -3 V
1 -3 V
2 -3 V
1 -3 V
2 -3 V
1 -3 V
2 -3 V
1 -3 V
1 -3 V
2 -3 V
1 -3 V
2 -2 V
stroke 1847 3498 M
1 -3 V
1 -3 V
2 -3 V
1 -3 V
2 -3 V
1 -3 V
1 -3 V
2 -3 V
1 -3 V
2 -3 V
1 -3 V
1 -2 V
2 -3 V
1 -3 V
2 -3 V
1 -3 V
2 -3 V
1 -3 V
1 -3 V
2 -3 V
1 -3 V
2 -3 V
1 -2 V
1 -3 V
2 -3 V
1 -3 V
2 -3 V
1 -3 V
1 -3 V
2 -3 V
1 -3 V
2 -3 V
1 -2 V
2 -3 V
1 -3 V
1 -3 V
2 -3 V
1 -3 V
2 -3 V
1 -3 V
1 -3 V
2 -3 V
1 -2 V
2 -3 V
1 -3 V
1 -3 V
2 -3 V
1 -3 V
2 -3 V
1 -3 V
1 -3 V
2 -3 V
1 -2 V
2 -3 V
1 -3 V
2 -3 V
1 -3 V
1 -3 V
2 -3 V
1 -3 V
2 -3 V
1 -2 V
1 -3 V
2 -3 V
1 -3 V
2 -3 V
1 -3 V
1 -3 V
2 -3 V
1 -3 V
2 -2 V
1 -3 V
1 -3 V
2 -3 V
1 -3 V
2 -3 V
1 -3 V
2 -3 V
1 -3 V
1 -2 V
2 -3 V
1 -3 V
2 -3 V
1 -3 V
1 -3 V
2 -3 V
1 -3 V
2 -2 V
1 -3 V
1 -3 V
2 -3 V
1 -3 V
2 -3 V
1 -3 V
2 -3 V
1 -2 V
1 -3 V
2 -3 V
1 -3 V
2 -3 V
1 -3 V
1 -3 V
2 -3 V
1 -2 V
stroke 1993 3197 M
2 -3 V
1 -3 V
1 -3 V
2 -3 V
1 -3 V
2 -3 V
1 -2 V
1 -3 V
2 -3 V
1 -3 V
2 -3 V
1 -3 V
2 -3 V
1 -2 V
1 -3 V
2 -3 V
1 -3 V
2 -3 V
1 -3 V
1 -3 V
2 -2 V
1 -3 V
2 -3 V
1 -3 V
1 -3 V
2 -3 V
1 -3 V
2 -2 V
1 -3 V
1 -3 V
2 -3 V
1 -3 V
2 -3 V
1 -3 V
2 -2 V
1 -3 V
1 -3 V
2 -3 V
1 -3 V
2 -3 V
1 -2 V
1 -3 V
2 -3 V
1 -3 V
2 -3 V
1 -3 V
1 -2 V
2 -3 V
1 -3 V
2 -3 V
1 -3 V
2 -3 V
1 -2 V
1 -3 V
2 -3 V
1 -3 V
2 -3 V
1 -3 V
1 -2 V
2 -3 V
1 -3 V
2 -3 V
1 -3 V
1 -3 V
2 -2 V
1 -3 V
2 -3 V
1 -3 V
1 -3 V
2 -3 V
1 -2 V
2 -3 V
1 -3 V
2 -3 V
1 -3 V
1 -2 V
2 -3 V
1 -3 V
2 -3 V
1 -3 V
1 -3 V
2 -2 V
1 -3 V
2 -3 V
1 -3 V
1 -3 V
2 -2 V
1 -3 V
2 -3 V
1 -3 V
2 -3 V
1 -2 V
1 -3 V
2 -3 V
1 -3 V
2 -3 V
1 -2 V
1 -3 V
2 -3 V
1 -3 V
2 -3 V
1 -2 V
1 -3 V
2 -3 V
stroke 2140 2902 M
1 -3 V
2 -3 V
1 -2 V
1 -3 V
2 -3 V
1 -3 V
2 -3 V
1 -2 V
2 -3 V
1 -3 V
1 -3 V
2 -2 V
1 -3 V
2 -3 V
1 -3 V
1 -3 V
2 -2 V
1 -3 V
2 -3 V
1 -3 V
1 -2 V
2 -3 V
1 -3 V
2 -3 V
1 -3 V
2 -2 V
1 -3 V
1 -3 V
2 -3 V
1 -2 V
2 -3 V
1 -3 V
1 -3 V
2 -2 V
1 -3 V
2 -3 V
1 -3 V
1 -3 V
2 -2 V
1 -3 V
2 -3 V
1 -3 V
1 -2 V
2 -3 V
1 -3 V
2 -3 V
1 -2 V
2 -3 V
1 -3 V
1 -3 V
2 -2 V
1 -3 V
2 -3 V
1 -3 V
1 -2 V
2 -3 V
1 -3 V
2 -3 V
1 -2 V
1 -3 V
2 -3 V
1 -3 V
2 -2 V
1 -3 V
1 -3 V
2 -2 V
1 -3 V
2 -3 V
1 -3 V
2 -2 V
1 -3 V
1 -3 V
2 -3 V
1 -2 V
2 -3 V
1 -3 V
1 -3 V
2 -2 V
1 -3 V
2 -3 V
1 -2 V
1 -3 V
2 -3 V
1 -3 V
2 -2 V
1 -3 V
2 -3 V
1 -2 V
1 -3 V
2 -3 V
1 -3 V
2 -2 V
1 -3 V
1 -3 V
2 -2 V
1 -3 V
2 -3 V
1 -2 V
1 -3 V
2 -3 V
1 -3 V
2 -2 V
1 -3 V
1 -3 V
stroke 2286 2616 M
2 -2 V
1 -3 V
2 -3 V
1 -2 V
2 -3 V
1 -3 V
1 -3 V
2 -2 V
1 -3 V
2 -3 V
1 -2 V
1 -3 V
2 -3 V
1 -2 V
2 -3 V
1 -3 V
1 -2 V
2 -3 V
1 -3 V
2 -2 V
1 -3 V
2 -3 V
1 -2 V
1 -3 V
2 -3 V
1 -2 V
2 -3 V
1 -3 V
1 -2 V
2 -3 V
1 -3 V
2 -2 V
1 -3 V
1 -3 V
2 -2 V
1 -3 V
2 -3 V
1 -2 V
1 -3 V
2 -3 V
1 -2 V
2 -3 V
1 -3 V
2 -2 V
1 -3 V
1 -3 V
2 -2 V
1 -3 V
2 -2 V
1 -3 V
1 -3 V
2 -2 V
1 -3 V
2 -3 V
1 -2 V
1 -3 V
2 -3 V
1 -2 V
2 -3 V
1 -2 V
1 -3 V
2 -3 V
1 -2 V
2 -3 V
1 -3 V
2 -2 V
1 -3 V
1 -2 V
2 -3 V
1 -3 V
2 -2 V
1 -3 V
1 -3 V
2 -2 V
1 -3 V
2 -2 V
1 -3 V
1 -3 V
2 -2 V
1 -3 V
2 -2 V
1 -3 V
2 -3 V
1 -2 V
1 -3 V
2 -2 V
1 -3 V
2 -3 V
1 -2 V
1 -3 V
2 -2 V
1 -3 V
2 -3 V
1 -2 V
1 -3 V
2 -2 V
1 -3 V
2 -3 V
1 -2 V
1 -3 V
2 -2 V
1 -3 V
2 -2 V
1 -3 V
stroke 2433 2342 M
2 -3 V
1 -2 V
1 -3 V
2 -2 V
1 -3 V
2 -2 V
1 -3 V
1 -3 V
2 -2 V
1 -3 V
2 -2 V
1 -3 V
1 -2 V
2 -3 V
1 -2 V
2 -3 V
1 -3 V
1 -2 V
2 -3 V
1 -2 V
2 -3 V
1 -2 V
2 -3 V
1 -2 V
1 -3 V
2 -3 V
1 -2 V
2 -3 V
1 -2 V
1 -3 V
2 -2 V
1 -3 V
2 -2 V
1 -3 V
1 -2 V
2 -3 V
1 -2 V
2 -3 V
1 -2 V
2 -3 V
1 -2 V
1 -3 V
2 -3 V
1 -2 V
2 -3 V
1 -2 V
1 -3 V
2 -2 V
1 -3 V
2 -2 V
1 -3 V
1 -2 V
2 -3 V
1 -2 V
2 -3 V
1 -2 V
1 -3 V
2 -2 V
1 -3 V
2 -2 V
1 -3 V
2 -2 V
1 -3 V
1 -2 V
2 -3 V
1 -2 V
2 -3 V
1 -2 V
1 -3 V
2 -2 V
1 -3 V
2 -2 V
1 -3 V
1 -2 V
2 -2 V
1 -3 V
2 -2 V
1 -3 V
2 -2 V
1 -3 V
1 -2 V
2 -3 V
1 -2 V
2 -3 V
1 -2 V
1 -3 V
2 -2 V
1 -3 V
2 -2 V
1 -2 V
1 -3 V
2 -2 V
1 -3 V
2 -2 V
1 -3 V
1 -2 V
2 -3 V
1 -2 V
2 -3 V
1 -2 V
2 -2 V
1 -3 V
1 -2 V
2 -3 V
stroke 2580 2081 M
1 -2 V
2 -3 V
1 -2 V
1 -2 V
2 -3 V
1 -2 V
2 -3 V
1 -2 V
1 -3 V
2 -2 V
1 -2 V
2 -3 V
1 -2 V
2 -3 V
1 -2 V
1 -2 V
2 -3 V
1 -2 V
2 -3 V
1 -2 V
1 -3 V
2 -2 V
1 -2 V
2 -3 V
1 -2 V
1 -3 V
2 -2 V
1 -2 V
2 -3 V
1 -2 V
1 -2 V
2 -3 V
1 -2 V
2 -3 V
1 -2 V
2 -2 V
1 -3 V
1 -2 V
2 -3 V
1 -2 V
2 -2 V
1 -3 V
1 -2 V
2 -2 V
1 -3 V
2 -2 V
1 -3 V
1 -2 V
2 -2 V
1 -3 V
2 -2 V
1 -2 V
1 -3 V
2 -2 V
1 -2 V
2 -3 V
1 -2 V
2 -2 V
1 -3 V
1 -2 V
2 -3 V
1 -2 V
2 -2 V
1 -3 V
1 -2 V
2 -2 V
1 -3 V
2 -2 V
1 -2 V
1 -3 V
2 -2 V
1 -2 V
2 -3 V
1 -2 V
2 -2 V
1 -3 V
1 -2 V
2 -2 V
1 -2 V
2 -3 V
1 -2 V
1 -2 V
2 -3 V
1 -2 V
2 -2 V
1 -3 V
1 -2 V
2 -2 V
1 -3 V
2 -2 V
1 -2 V
1 -2 V
2 -3 V
1 -2 V
2 -2 V
1 -3 V
2 -2 V
1 -2 V
1 -2 V
2 -3 V
1 -2 V
2 -2 V
1 -3 V
1 -2 V
stroke 2726 1836 M
2 -2 V
1 -2 V
2 -3 V
1 -2 V
1 -2 V
2 -3 V
1 -2 V
2 -2 V
1 -2 V
2 -3 V
1 -2 V
1 -2 V
2 -2 V
1 -3 V
2 -2 V
1 -2 V
1 -2 V
2 -3 V
1 -2 V
2 -2 V
1 -2 V
1 -3 V
2 -2 V
1 -2 V
2 -2 V
1 -3 V
1 -2 V
2 -2 V
1 -2 V
2 -2 V
1 -3 V
2 -2 V
1 -2 V
1 -2 V
2 -3 V
1 -2 V
2 -2 V
1 -2 V
1 -2 V
2 -3 V
1 -2 V
2 -2 V
1 -2 V
1 -3 V
2 -2 V
1 -2 V
2 -2 V
1 -2 V
1 -3 V
2 -2 V
1 -2 V
2 -2 V
1 -2 V
2 -2 V
1 -3 V
1 -2 V
2 -2 V
1 -2 V
2 -2 V
1 -3 V
1 -2 V
2 -2 V
1 -2 V
2 -2 V
1 -2 V
1 -3 V
2 -2 V
1 -2 V
2 -2 V
1 -2 V
2 -2 V
1 -3 V
1 -2 V
2 -2 V
1 -2 V
2 -2 V
1 -2 V
1 -3 V
2 -2 V
1 -2 V
2 -2 V
1 -2 V
1 -2 V
2 -2 V
1 -3 V
2 -2 V
1 -2 V
1 -2 V
2 -2 V
1 -2 V
2 -2 V
1 -2 V
2 -3 V
1 -2 V
1 -2 V
2 -2 V
1 -2 V
2 -2 V
1 -2 V
1 -2 V
2 -3 V
1 -2 V
2 -2 V
1 -2 V
stroke 2873 1608 M
1 -2 V
2 -2 V
1 -2 V
2 -2 V
1 -2 V
2 -2 V
1 -3 V
1 -2 V
2 -2 V
1 -2 V
2 -2 V
1 -2 V
1 -2 V
2 -2 V
1 -2 V
2 -2 V
1 -2 V
1 -2 V
2 -2 V
1 -3 V
2 -2 V
1 -2 V
1 -2 V
2 -2 V
1 -2 V
2 -2 V
1 -2 V
2 -2 V
1 -2 V
1 -2 V
2 -2 V
1 -2 V
2 -2 V
1 -2 V
1 -2 V
2 -2 V
1 -2 V
2 -3 V
1 -2 V
1 -2 V
2 -2 V
1 -2 V
2 -2 V
1 -2 V
1 -2 V
2 -2 V
1 -2 V
2 -2 V
1 -2 V
2 -2 V
1 -2 V
1 -2 V
2 -2 V
1 -2 V
2 -2 V
1 -2 V
1 -2 V
2 -2 V
1 -2 V
2 -2 V
1 -2 V
1 -2 V
2 -2 V
1 -2 V
2 -2 V
1 -2 V
2 -2 V
1 -2 V
1 -2 V
2 -2 V
1 -2 V
2 -2 V
1 -2 V
1 -2 V
2 -2 V
1 -2 V
2 -2 V
1 -2 V
1 -2 V
2 -2 V
1 -1 V
2 -2 V
1 -2 V
1 -2 V
2 -2 V
1 -2 V
2 -2 V
1 -2 V
2 -2 V
1 -2 V
1 -2 V
2 -2 V
1 -2 V
2 -2 V
1 -2 V
1 -2 V
2 -2 V
1 -2 V
2 -1 V
1 -2 V
1 -2 V
2 -2 V
1 -2 V
2 -2 V
stroke 3020 1399 M
1 -2 V
2 -2 V
1 -2 V
1 -2 V
2 -2 V
1 -2 V
2 -1 V
1 -2 V
1 -2 V
2 -2 V
1 -2 V
2 -2 V
1 -2 V
1 -2 V
2 -2 V
1 -2 V
2 -1 V
1 -2 V
1 -2 V
2 -2 V
1 -2 V
2 -2 V
1 -2 V
2 -2 V
1 -1 V
1 -2 V
2 -2 V
1 -2 V
2 -2 V
1 -2 V
1 -2 V
2 -1 V
1 -2 V
2 -2 V
1 -2 V
1 -2 V
2 -2 V
1 -2 V
2 -1 V
1 -2 V
1 -2 V
2 -2 V
1 -2 V
2 -2 V
1 -1 V
2 -2 V
1 -2 V
1 -2 V
2 -2 V
1 -2 V
2 -1 V
1 -2 V
1 -2 V
2 -2 V
1 -2 V
2 -2 V
1 -1 V
1 -2 V
2 -2 V
1 -2 V
2 -2 V
1 -1 V
2 -2 V
1 -2 V
1 -2 V
2 -2 V
1 -1 V
2 -2 V
1 -2 V
1 -2 V
2 -2 V
1 -1 V
2 -2 V
1 -2 V
1 -2 V
2 -1 V
1 -2 V
2 -2 V
1 -2 V
1 -2 V
2 -1 V
1 -2 V
2 -2 V
1 -2 V
2 -1 V
1 -2 V
1 -2 V
2 -2 V
1 -1 V
2 -2 V
1 -2 V
1 -2 V
2 -1 V
1 -2 V
2 -2 V
1 -2 V
1 -1 V
2 -2 V
1 -2 V
2 -1 V
1 -2 V
1 -2 V
2 -2 V
1 -1 V
stroke 3166 1210 M
2 -2 V
1 -2 V
2 -2 V
1 -1 V
1 -2 V
2 -2 V
1 -1 V
2 -2 V
1 -2 V
1 -2 V
2 -1 V
1 -2 V
2 -2 V
1 -1 V
1 -2 V
2 -2 V
1 -1 V
2 -2 V
1 -2 V
2 -1 V
1 -2 V
1 -2 V
2 -1 V
1 -2 V
2 -2 V
1 -1 V
1 -2 V
2 -2 V
1 -1 V
2 -2 V
1 -2 V
1 -1 V
2 -2 V
1 -2 V
2 -1 V
1 -2 V
1 -2 V
2 -1 V
1 -2 V
2 -2 V
1 -1 V
2 -2 V
1 -2 V
1 -1 V
2 -2 V
1 -2 V
2 -1 V
1 -2 V
1 -1 V
2 -2 V
1 -2 V
2 -1 V
1 -2 V
1 -2 V
2 -1 V
1 -2 V
2 -1 V
1 -2 V
2 -2 V
1 -1 V
1 -2 V
2 -1 V
1 -2 V
2 -2 V
1 -1 V
1 -2 V
2 -1 V
1 -2 V
2 -2 V
1 -1 V
1 -2 V
2 -1 V
1 -2 V
2 -1 V
1 -2 V
1 -2 V
2 -1 V
1 -2 V
2 -1 V
1 -2 V
2 -1 V
1 -2 V
1 -2 V
2 -1 V
1 -2 V
2 -1 V
1 -2 V
1 -1 V
2 -2 V
1 -1 V
2 -2 V
1 -2 V
1 -1 V
2 -2 V
1 -1 V
2 -2 V
1 -1 V
1 -2 V
2 -1 V
1 -2 V
2 -1 V
1 -2 V
2 -1 V
1 -2 V
stroke 3313 1041 M
1 -1 V
2 -2 V
1 -1 V
2 -2 V
1 -1 V
1 -2 V
2 -1 V
1 -2 V
2 -1 V
1 -2 V
1 -1 V
2 -2 V
1 -1 V
2 -2 V
1 -1 V
2 -2 V
1 -1 V
1 -2 V
2 -1 V
1 -2 V
2 -1 V
1 -2 V
1 -1 V
2 -2 V
1 -1 V
2 -2 V
1 -1 V
1 -2 V
2 -1 V
1 -1 V
2 -2 V
1 -1 V
1 -2 V
2 -1 V
1 -2 V
2 -1 V
1 -2 V
2 -1 V
1 -1 V
1 -2 V
2 -1 V
1 -2 V
2 -1 V
1 -2 V
1 -1 V
2 -2 V
1 -1 V
2 -1 V
1 -2 V
1 -1 V
2 -2 V
1 -1 V
2 -1 V
1 -2 V
2 -1 V
1 -2 V
1 -1 V
2 -1 V
1 -2 V
2 -1 V
1 -2 V
1 -1 V
2 -1 V
1 -2 V
2 -1 V
1 -2 V
1 -1 V
2 -1 V
1 -2 V
2 -1 V
1 -1 V
1 -2 V
2 -1 V
1 -2 V
2 -1 V
1 -1 V
2 -2 V
1 -1 V
1 -1 V
2 -2 V
1 -1 V
2 -1 V
1 -2 V
1 -1 V
2 -2 V
1 -1 V
2 -1 V
1 -2 V
1 -1 V
2 -1 V
1 -2 V
2 -1 V
1 -1 V
2 -2 V
1 -1 V
1 -1 V
2 -2 V
1 -1 V
2 -1 V
1 -1 V
1 -2 V
2 -1 V
1 -1 V
2 -2 V
stroke 3460 894 M
1 -1 V
1 -1 V
2 -2 V
1 -1 V
2 -1 V
1 -2 V
1 -1 V
2 -1 V
1 -1 V
2 -2 V
1 -1 V
2 -1 V
1 -2 V
1 -1 V
2 -1 V
1 -1 V
2 -2 V
1 -1 V
1 -1 V
2 -1 V
1 -2 V
2 -1 V
1 -1 V
1 -2 V
2 -1 V
1 -1 V
2 -1 V
1 -2 V
1 -1 V
2 -1 V
1 -1 V
2 -1 V
1 -2 V
2 -1 V
1 -1 V
1 -1 V
2 -2 V
1 -1 V
2 -1 V
1 -1 V
1 -2 V
2 -1 V
1 -1 V
2 -1 V
1 -1 V
1 -2 V
2 -1 V
1 -1 V
2 -1 V
1 -2 V
2 -1 V
1 -1 V
1 -1 V
2 -1 V
1 -2 V
2 -1 V
1 -1 V
1 -1 V
2 -1 V
1 -1 V
2 -2 V
1 -1 V
1 -1 V
2 -1 V
1 -1 V
2 -2 V
1 -1 V
1 -1 V
2 -1 V
1 -1 V
2 -1 V
1 -2 V
2 -1 V
1 -1 V
1 -1 V
2 -1 V
1 -1 V
2 -1 V
1 -2 V
1 -1 V
2 -1 V
1 -1 V
2 -1 V
1 -1 V
1 -1 V
2 -2 V
1 -1 V
2 -1 V
1 -1 V
2 -1 V
1 -1 V
1 -1 V
2 -1 V
1 -2 V
2 -1 V
1 -1 V
1 -1 V
2 -1 V
1 -1 V
2 -1 V
1 -1 V
1 -1 V
2 -1 V
1 -2 V
stroke 3606 769 M
2 -1 V
1 -1 V
1 -1 V
2 -1 V
1 -1 V
2 -1 V
1 -1 V
2 -1 V
1 -1 V
1 -1 V
2 -1 V
1 -2 V
2 -1 V
1 -1 V
1 -1 V
2 -1 V
1 -1 V
2 -1 V
1 -1 V
1 -1 V
2 -1 V
1 -1 V
2 -1 V
1 -1 V
1 -1 V
2 -1 V
1 -1 V
2 -1 V
1 -1 V
2 -1 V
1 -1 V
1 -1 V
2 -2 V
1 -1 V
2 -1 V
1 -1 V
1 -1 V
2 -1 V
1 -1 V
2 -1 V
1 -1 V
1 -1 V
2 -1 V
1 -1 V
2 -1 V
1 -1 V
2 -1 V
1 -1 V
1 -1 V
2 -1 V
1 -1 V
2 -1 V
1 -1 V
1 -1 V
2 -1 V
1 -1 V
2 -1 V
1 -1 V
1 0 V
2 -1 V
1 -1 V
2 -1 V
1 -1 V
1 -1 V
2 -1 V
1 -1 V
2 -1 V
1 -1 V
2 -1 V
1 -1 V
1 -1 V
2 -1 V
1 -1 V
2 -1 V
1 -1 V
1 -1 V
2 -1 V
1 -1 V
2 0 V
1 -1 V
1 -1 V
2 -1 V
1 -1 V
2 -1 V
1 -1 V
2 -1 V
1 -1 V
1 -1 V
2 -1 V
1 -1 V
2 -1 V
1 0 V
1 -1 V
2 -1 V
1 -1 V
2 -1 V
1 -1 V
1 -1 V
2 -1 V
1 -1 V
2 0 V
1 -1 V
1 -1 V
2 -1 V
stroke 3753 667 M
1 -1 V
2 -1 V
1 -1 V
2 -1 V
1 0 V
1 -1 V
2 -1 V
1 -1 V
2 -1 V
1 -1 V
1 -1 V
2 -1 V
1 0 V
2 -1 V
1 -1 V
1 -1 V
2 -1 V
1 -1 V
2 0 V
1 -1 V
1 -1 V
2 -1 V
1 -1 V
2 -1 V
1 0 V
2 -1 V
1 -1 V
1 -1 V
2 -1 V
1 -1 V
2 0 V
1 -1 V
1 -1 V
2 -1 V
1 -1 V
2 0 V
1 -1 V
1 -1 V
2 -1 V
1 -1 V
2 0 V
1 -1 V
2 -1 V
1 -1 V
1 -1 V
2 0 V
1 -1 V
2 -1 V
1 -1 V
1 0 V
2 -1 V
1 -1 V
2 -1 V
1 -1 V
1 0 V
2 -1 V
1 -1 V
2 -1 V
1 0 V
1 -1 V
2 -1 V
1 -1 V
2 0 V
1 -1 V
2 -1 V
1 -1 V
1 0 V
2 -1 V
1 -1 V
2 -1 V
1 0 V
1 -1 V
2 -1 V
1 0 V
2 -1 V
1 -1 V
1 -1 V
2 0 V
1 -1 V
2 -1 V
1 -1 V
1 0 V
2 -1 V
1 -1 V
2 0 V
1 -1 V
2 -1 V
1 0 V
1 -1 V
2 -1 V
1 -1 V
2 0 V
1 -1 V
1 -1 V
2 0 V
1 -1 V
2 -1 V
1 0 V
1 -1 V
2 -1 V
1 0 V
2 -1 V
1 -1 V
2 0 V
stroke 3900 587 M
1 -1 V
1 -1 V
2 0 V
1 -1 V
2 -1 V
1 0 V
1 -1 V
2 -1 V
1 0 V
2 -1 V
1 -1 V
1 0 V
2 -1 V
1 0 V
2 -1 V
1 -1 V
1 0 V
2 -1 V
1 -1 V
2 0 V
1 -1 V
2 -1 V
1 0 V
1 -1 V
2 0 V
1 -1 V
2 -1 V
1 0 V
1 -1 V
2 0 V
1 -1 V
2 -1 V
1 0 V
1 -1 V
2 0 V
1 -1 V
2 -1 V
1 0 V
2 -1 V
1 0 V
1 -1 V
2 -1 V
1 0 V
2 -1 V
1 0 V
1 -1 V
2 0 V
1 -1 V
2 -1 V
1 0 V
1 -1 V
2 0 V
1 -1 V
2 0 V
1 -1 V
1 0 V
2 -1 V
1 -1 V
2 0 V
1 -1 V
2 0 V
1 -1 V
1 0 V
2 -1 V
1 0 V
2 -1 V
1 0 V
1 -1 V
2 0 V
1 -1 V
2 0 V
1 -1 V
1 -1 V
2 0 V
1 -1 V
2 0 V
1 -1 V
1 0 V
2 -1 V
1 0 V
2 -1 V
1 0 V
2 -1 V
1 0 V
1 -1 V
2 0 V
1 -1 V
2 0 V
1 -1 V
1 0 V
2 0 V
1 -1 V
2 0 V
1 -1 V
1 0 V
2 -1 V
1 0 V
2 -1 V
1 0 V
2 -1 V
1 0 V
1 -1 V
2 0 V
1 -1 V
stroke 4046 528 M
2 0 V
1 0 V
1 -1 V
2 0 V
1 -1 V
2 0 V
1 -1 V
1 0 V
2 -1 V
1 0 V
2 0 V
1 -1 V
1 0 V
2 -1 V
1 0 V
2 -1 V
1 0 V
2 0 V
1 -1 V
1 0 V
2 -1 V
1 0 V
2 -1 V
1 0 V
1 0 V
2 -1 V
1 0 V
2 -1 V
1 0 V
1 0 V
2 -1 V
1 0 V
2 0 V
1 -1 V
2 0 V
1 -1 V
1 0 V
2 0 V
1 -1 V
2 0 V
1 -1 V
1 0 V
2 0 V
1 -1 V
2 0 V
1 0 V
1 -1 V
2 0 V
1 0 V
2 -1 V
1 0 V
1 -1 V
2 0 V
1 0 V
2 -1 V
1 0 V
2 0 V
1 -1 V
1 0 V
2 0 V
1 -1 V
2 0 V
1 0 V
1 -1 V
2 0 V
1 0 V
2 -1 V
1 0 V
1 0 V
2 -1 V
1 0 V
2 0 V
1 0 V
2 -1 V
1 0 V
1 0 V
2 -1 V
1 0 V
2 0 V
1 -1 V
1 0 V
2 0 V
1 0 V
2 -1 V
1 0 V
1 0 V
2 -1 V
1 0 V
2 0 V
1 0 V
1 -1 V
2 0 V
1 0 V
2 -1 V
1 0 V
2 0 V
1 0 V
1 -1 V
2 0 V
1 0 V
2 0 V
1 -1 V
1 0 V
2 0 V
stroke 4193 492 M
1 0 V
2 -1 V
1 0 V
1 0 V
2 0 V
1 -1 V
2 0 V
1 0 V
1 0 V
2 -1 V
1 0 V
2 0 V
1 0 V
2 0 V
1 -1 V
1 0 V
2 0 V
1 0 V
2 -1 V
1 0 V
1 0 V
2 0 V
1 0 V
2 -1 V
1 0 V
1 0 V
2 0 V
1 0 V
2 -1 V
1 0 V
2 0 V
1 0 V
1 0 V
2 0 V
1 -1 V
2 0 V
1 0 V
1 0 V
2 0 V
1 -1 V
2 0 V
1 0 V
1 0 V
2 0 V
1 0 V
2 -1 V
1 0 V
1 0 V
2 0 V
1 0 V
2 0 V
1 0 V
2 -1 V
1 0 V
1 0 V
2 0 V
1 0 V
2 0 V
1 0 V
1 -1 V
2 0 V
1 0 V
2 0 V
1 0 V
1 0 V
2 0 V
1 0 V
2 -1 V
1 0 V
2 0 V
1 0 V
1 0 V
2 0 V
1 0 V
2 0 V
1 0 V
1 -1 V
2 0 V
1 0 V
2 0 V
1 0 V
1 0 V
2 0 V
1 0 V
2 0 V
1 0 V
1 0 V
2 -1 V
1 0 V
2 0 V
1 0 V
2 0 V
1 0 V
1 0 V
2 0 V
1 0 V
2 0 V
1 0 V
1 0 V
2 0 V
1 0 V
2 -1 V
1 0 V
1 0 V
stroke 4339 476 M
2 0 V
1 0 V
2 0 V
1 0 V
1 0 V
2 0 V
1 0 V
2 0 V
1 0 V
2 0 V
1 0 V
1 0 V
2 0 V
1 0 V
2 0 V
1 0 V
1 0 V
2 0 V
1 0 V
2 0 V
1 0 V
1 0 V
2 0 V
1 0 V
2 0 V
1 0 V
2 0 V
1 0 V
1 0 V
2 0 V
1 0 V
2 0 V
1 0 V
1 0 V
2 0 V
1 0 V
2 0 V
1 0 V
1 0 V
2 0 V
1 0 V
2 0 V
1 0 V
1 0 V
2 0 V
1 0 V
2 0 V
1 0 V
2 0 V
1 0 V
1 0 V
2 0 V
1 0 V
2 0 V
1 0 V
1 0 V
2 1 V
1 0 V
2 0 V
1 0 V
1 0 V
2 0 V
1 0 V
2 0 V
1 0 V
2 0 V
1 0 V
1 0 V
2 0 V
1 0 V
2 0 V
1 1 V
1 0 V
2 0 V
1 0 V
2 0 V
1 0 V
1 0 V
2 0 V
1 0 V
2 0 V
1 0 V
1 1 V
2 0 V
1 0 V
2 0 V
1 0 V
2 0 V
1 0 V
1 0 V
2 0 V
1 1 V
2 0 V
1 0 V
1 0 V
2 0 V
1 0 V
2 0 V
1 0 V
1 1 V
2 0 V
1 0 V
2 0 V
1 0 V
stroke 4486 481 M
1 0 V
2 0 V
1 1 V
2 0 V
1 0 V
2 0 V
1 0 V
1 0 V
2 0 V
1 1 V
2 0 V
1 0 V
1 0 V
2 0 V
1 0 V
2 1 V
1 0 V
1 0 V
2 0 V
1 0 V
2 0 V
1 1 V
2 0 V
1 0 V
1 0 V
2 0 V
1 1 V
2 0 V
1 0 V
1 0 V
2 0 V
1 1 V
2 0 V
1 0 V
1 0 V
2 0 V
1 1 V
2 0 V
1 0 V
1 0 V
2 0 V
1 1 V
2 0 V
1 0 V
2 0 V
1 0 V
1 1 V
2 0 V
1 0 V
2 0 V
1 1 V
1 0 V
2 0 V
1 0 V
2 1 V
1 0 V
1 0 V
2 0 V
1 0 V
2 1 V
1 0 V
2 0 V
1 0 V
1 1 V
2 0 V
1 0 V
2 0 V
1 1 V
1 0 V
2 0 V
1 1 V
2 0 V
1 0 V
1 0 V
2 1 V
1 0 V
2 0 V
1 0 V
1 1 V
2 0 V
1 0 V
2 1 V
1 0 V
2 0 V
1 0 V
1 1 V
2 0 V
1 0 V
2 1 V
1 0 V
1 0 V
2 0 V
1 1 V
2 0 V
1 0 V
1 1 V
2 0 V
1 0 V
2 1 V
1 0 V
1 0 V
2 1 V
1 0 V
2 0 V
stroke 4633 505 M
1 1 V
2 0 V
1 0 V
1 0 V
2 1 V
1 0 V
2 0 V
1 1 V
1 0 V
2 0 V
1 1 V
2 0 V
1 1 V
1 0 V
2 0 V
1 1 V
2 0 V
1 0 V
2 1 V
1 0 V
1 0 V
2 1 V
1 0 V
2 0 V
1 1 V
1 0 V
2 0 V
1 1 V
2 0 V
1 1 V
1 0 V
2 0 V
1 1 V
2 0 V
1 0 V
1 1 V
2 0 V
1 1 V
2 0 V
1 0 V
2 1 V
1 0 V
1 1 V
2 0 V
1 0 V
2 1 V
1 0 V
1 1 V
2 0 V
1 0 V
2 1 V
1 0 V
1 1 V
2 0 V
1 0 V
2 1 V
1 0 V
1 1 V
2 0 V
1 0 V
2 1 V
1 0 V
2 1 V
1 0 V
1 1 V
2 0 V
1 0 V
2 1 V
1 0 V
1 1 V
2 0 V
1 1 V
2 0 V
1 0 V
1 1 V
2 0 V
1 1 V
2 0 V
1 1 V
2 0 V
1 1 V
1 0 V
2 1 V
1 0 V
2 0 V
1 1 V
1 0 V
2 1 V
1 0 V
2 1 V
1 0 V
1 1 V
2 0 V
1 1 V
2 0 V
1 1 V
1 0 V
2 1 V
1 0 V
2 1 V
1 0 V
2 0 V
1 1 V
1 0 V
stroke 4779 547 M
2 1 V
1 0 V
2 1 V
1 0 V
1 1 V
2 0 V
1 1 V
2 0 V
1 1 V
1 0 V
2 1 V
1 0 V
2 1 V
1 0 V
2 1 V
1 1 V
1 0 V
2 1 V
1 0 V
2 1 V
1 0 V
1 1 V
2 0 V
1 1 V
2 0 V
1 1 V
1 0 V
2 1 V
1 0 V
2 1 V
1 0 V
1 1 V
2 0 V
1 1 V
2 1 V
1 0 V
2 1 V
1 0 V
1 1 V
2 0 V
1 1 V
2 0 V
1 1 V
1 1 V
2 0 V
1 1 V
2 0 V
1 1 V
1 0 V
2 1 V
1 0 V
2 1 V
1 1 V
2 0 V
1 1 V
1 0 V
2 1 V
1 0 V
2 1 V
1 1 V
1 0 V
2 1 V
1 0 V
2 1 V
1 1 V
1 0 V
2 1 V
1 0 V
2 1 V
1 1 V
1 0 V
2 1 V
1 0 V
2 1 V
1 1 V
2 0 V
1 1 V
1 0 V
2 1 V
1 1 V
2 0 V
1 1 V
1 0 V
2 1 V
1 1 V
2 0 V
1 1 V
1 1 V
2 0 V
1 1 V
2 0 V
1 1 V
1 1 V
2 0 V
1 1 V
2 1 V
1 0 V
2 1 V
1 0 V
1 1 V
2 1 V
1 0 V
2 1 V
1 1 V
stroke 4926 607 M
1 0 V
2 1 V
1 1 V
2 0 V
1 1 V
1 1 V
2 0 V
1 1 V
2 1 V
1 0 V
2 1 V
1 1 V
1 0 V
2 1 V
1 1 V
2 0 V
1 1 V
1 1 V
2 0 V
1 1 V
2 1 V
1 0 V
1 1 V
2 1 V
1 0 V
2 1 V
1 1 V
1 0 V
2 1 V
1 1 V
2 0 V
1 1 V
2 1 V
1 1 V
1 0 V
2 1 V
1 1 V
2 0 V
1 1 V
1 1 V
2 0 V
1 1 V
2 1 V
1 1 V
1 0 V
2 1 V
1 1 V
2 0 V
1 1 V
2 1 V
1 1 V
1 0 V
2 1 V
1 1 V
2 0 V
1 1 V
1 1 V
2 1 V
1 0 V
2 1 V
1 1 V
1 1 V
2 0 V
1 1 V
2 1 V
1 1 V
1 0 V
2 1 V
1 1 V
2 0 V
1 1 V
2 1 V
1 1 V
1 0 V
2 1 V
1 1 V
2 1 V
1 0 V
1 1 V
2 1 V
1 1 V
2 1 V
1 0 V
1 1 V
2 1 V
1 1 V
2 0 V
1 1 V
1 1 V
2 1 V
1 0 V
2 1 V
1 1 V
2 1 V
1 1 V
1 0 V
2 1 V
1 1 V
2 1 V
1 0 V
1 1 V
2 1 V
1 1 V
2 1 V
stroke 5073 682 M
1 0 V
1 1 V
2 1 V
1 1 V
2 1 V
1 0 V
2 1 V
1 1 V
1 1 V
2 1 V
1 0 V
2 1 V
1 1 V
1 1 V
2 1 V
1 0 V
2 1 V
1 1 V
1 1 V
2 1 V
1 0 V
2 1 V
1 1 V
1 1 V
2 1 V
1 1 V
2 0 V
1 1 V
2 1 V
1 1 V
1 1 V
2 1 V
1 0 V
2 1 V
1 1 V
1 1 V
2 1 V
1 1 V
2 0 V
1 1 V
1 1 V
2 1 V
1 1 V
2 1 V
1 0 V
2 1 V
1 1 V
1 1 V
2 1 V
1 1 V
2 1 V
1 0 V
1 1 V
2 1 V
1 1 V
2 1 V
1 1 V
1 1 V
2 0 V
1 1 V
2 1 V
1 1 V
1 1 V
2 1 V
1 1 V
2 1 V
1 0 V
2 1 V
1 1 V
1 1 V
2 1 V
1 1 V
2 1 V
1 1 V
1 0 V
2 1 V
1 1 V
2 1 V
1 1 V
1 1 V
2 1 V
1 1 V
2 1 V
1 0 V
1 1 V
2 1 V
1 1 V
2 1 V
1 1 V
2 1 V
1 1 V
1 1 V
2 1 V
1 1 V
2 0 V
1 1 V
1 1 V
2 1 V
1 1 V
2 1 V
1 1 V
1 1 V
2 1 V
1 1 V
stroke 5219 771 M
2 1 V
1 0 V
2 1 V
1 1 V
1 1 V
2 1 V
1 1 V
2 1 V
1 1 V
1 1 V
2 1 V
1 1 V
2 1 V
1 1 V
1 1 V
2 1 V
1 0 V
2 1 V
1 1 V
1 1 V
2 1 V
1 1 V
2 1 V
1 1 V
2 1 V
1 1 V
1 1 V
2 1 V
1 1 V
2 1 V
1 1 V
1 1 V
2 1 V
1 1 V
2 1 V
1 1 V
1 0 V
2 1 V
1 1 V
2 1 V
1 1 V
2 1 V
1 1 V
1 1 V
2 1 V
1 1 V
2 1 V
1 1 V
1 1 V
2 1 V
1 1 V
2 1 V
1 1 V
1 1 V
2 1 V
1 1 V
2 1 V
1 1 V
1 1 V
2 1 V
1 1 V
2 1 V
1 1 V
2 1 V
1 1 V
1 1 V
2 1 V
1 1 V
2 1 V
1 1 V
1 1 V
2 1 V
1 1 V
2 1 V
1 1 V
1 1 V
2 1 V
1 1 V
2 1 V
1 1 V
1 1 V
2 1 V
1 1 V
2 1 V
1 1 V
2 1 V
1 1 V
1 1 V
2 1 V
1 1 V
2 1 V
1 1 V
1 1 V
2 1 V
1 1 V
2 1 V
1 1 V
1 1 V
2 1 V
1 1 V
2 1 V
1 1 V
2 1 V
1 1 V
stroke 5366 872 M
1 1 V
2 1 V
1 1 V
2 2 V
1 1 V
1 1 V
2 1 V
1 1 V
2 1 V
1 1 V
1 1 V
2 1 V
1 1 V
2 1 V
1 1 V
1 1 V
2 1 V
1 1 V
2 1 V
1 1 V
2 1 V
1 1 V
1 1 V
2 1 V
1 2 V
2 1 V
1 1 V
1 1 V
2 1 V
1 1 V
2 1 V
1 1 V
1 1 V
2 1 V
1 1 V
2 1 V
1 1 V
1 1 V
2 1 V
1 2 V
2 1 V
1 1 V
2 1 V
1 1 V
1 1 V
2 1 V
1 1 V
2 1 V
1 1 V
1 1 V
2 1 V
1 1 V
2 2 V
1 1 V
1 1 V
2 1 V
1 1 V
2 1 V
1 1 V
2 1 V
1 1 V
1 1 V
2 1 V
1 2 V
2 1 V
1 1 V
1 1 V
2 1 V
1 1 V
2 1 V
1 1 V
1 1 V
2 1 V
1 2 V
2 1 V
1 1 V
1 1 V
2 1 V
1 1 V
2 1 V
1 1 V
2 1 V
1 2 V
1 1 V
2 1 V
1 1 V
2 1 V
1 1 V
1 1 V
2 1 V
1 2 V
2 1 V
1 1 V
1 1 V
2 1 V
1 1 V
2 1 V
1 1 V
2 2 V
1 1 V
1 1 V
2 1 V
1 1 V
2 1 V
stroke 5513 985 M
1 1 V
1 1 V
2 2 V
1 1 V
2 1 V
1 1 V
1 1 V
2 1 V
1 1 V
2 2 V
1 1 V
1 1 V
2 1 V
1 1 V
2 1 V
1 1 V
2 2 V
1 1 V
1 1 V
2 1 V
1 1 V
2 1 V
1 1 V
1 2 V
2 1 V
1 1 V
2 1 V
1 1 V
1 1 V
2 2 V
1 1 V
2 1 V
1 1 V
2 1 V
1 1 V
1 2 V
2 1 V
1 1 V
2 1 V
1 1 V
1 1 V
2 2 V
1 1 V
2 1 V
1 1 V
1 1 V
2 1 V
1 2 V
2 1 V
1 1 V
1 1 V
2 1 V
1 1 V
2 2 V
1 1 V
2 1 V
1 1 V
1 1 V
2 1 V
1 2 V
2 1 V
1 1 V
1 1 V
2 1 V
1 2 V
2 1 V
1 1 V
1 1 V
2 1 V
1 2 V
2 1 V
1 1 V
1 1 V
2 1 V
1 1 V
2 2 V
1 1 V
2 1 V
1 1 V
1 1 V
2 2 V
1 1 V
2 1 V
1 1 V
1 1 V
2 2 V
1 1 V
2 1 V
1 1 V
1 1 V
2 2 V
1 1 V
2 1 V
1 1 V
2 2 V
1 1 V
1 1 V
2 1 V
1 1 V
2 2 V
1 1 V
1 1 V
2 1 V
1 1 V
stroke 5659 1107 M
2 2 V
1 1 V
1 1 V
2 1 V
1 2 V
2 1 V
1 1 V
1 1 V
2 1 V
1 2 V
2 1 V
1 1 V
2 1 V
1 2 V
1 1 V
2 1 V
1 1 V
2 1 V
1 2 V
1 1 V
2 1 V
1 1 V
2 2 V
1 1 V
1 1 V
2 1 V
1 2 V
2 1 V
1 1 V
2 1 V
1 1 V
1 2 V
2 1 V
1 1 V
2 1 V
1 2 V
1 1 V
2 1 V
1 1 V
2 2 V
1 1 V
1 1 V
2 1 V
1 2 V
2 1 V
1 1 V
1 1 V
2 2 V
1 1 V
2 1 V
1 1 V
2 2 V
1 1 V
1 1 V
2 1 V
1 2 V
2 1 V
1 1 V
1 1 V
2 2 V
1 1 V
2 1 V
1 1 V
1 2 V
2 1 V
1 1 V
2 1 V
1 2 V
1 1 V
2 1 V
1 1 V
2 2 V
1 1 V
2 1 V
1 2 V
1 1 V
2 1 V
1 1 V
2 2 V
1 1 V
1 1 V
2 1 V
1 2 V
2 1 V
1 1 V
1 2 V
2 1 V
1 1 V
2 1 V
1 2 V
2 1 V
1 1 V
1 1 V
2 2 V
1 1 V
2 1 V
1 2 V
1 1 V
2 1 V
1 1 V
2 2 V
1 1 V
1 1 V
2 1 V
stroke 5806 1237 M
1 2 V
2 1 V
1 1 V
1 2 V
2 1 V
1 1 V
2 1 V
1 2 V
2 1 V
1 1 V
1 2 V
2 1 V
1 1 V
2 2 V
1 1 V
1 1 V
2 1 V
1 2 V
2 1 V
1 1 V
1 2 V
2 1 V
1 1 V
2 1 V
1 2 V
2 1 V
1 1 V
1 2 V
2 1 V
1 1 V
2 2 V
1 1 V
1 1 V
2 1 V
1 2 V
2 1 V
1 1 V
1 2 V
2 1 V
1 1 V
2 2 V
1 1 V
1 1 V
2 1 V
1 2 V
2 1 V
1 1 V
2 2 V
1 1 V
1 1 V
2 2 V
1 1 V
2 1 V
1 2 V
1 1 V
2 1 V
1 1 V
2 2 V
1 1 V
1 1 V
2 2 V
1 1 V
2 1 V
1 2 V
1 1 V
2 1 V
1 2 V
2 1 V
1 1 V
2 2 V
1 1 V
1 1 V
2 2 V
1 1 V
2 1 V
1 2 V
1 1 V
2 1 V
1 1 V
2 2 V
1 1 V
1 1 V
2 2 V
1 1 V
2 1 V
1 2 V
2 1 V
1 1 V
1 2 V
2 1 V
1 1 V
2 2 V
1 1 V
1 1 V
2 2 V
1 1 V
2 1 V
1 2 V
1 1 V
2 1 V
1 2 V
2 1 V
1 1 V
1 2 V
stroke 5952 1374 M
2 1 V
1 1 V
2 2 V
1 1 V
2 1 V
1 2 V
1 1 V
2 1 V
1 2 V
2 1 V
1 1 V
1 2 V
2 1 V
1 1 V
2 2 V
1 1 V
1 1 V
2 2 V
1 1 V
2 1 V
1 2 V
1 1 V
2 1 V
1 2 V
2 1 V
1 1 V
2 2 V
1 1 V
1 2 V
2 1 V
1 1 V
2 2 V
1 1 V
1 1 V
2 2 V
1 1 V
2 1 V
1 2 V
1 1 V
2 1 V
1 2 V
2 1 V
1 1 V
2 2 V
1 1 V
1 1 V
2 2 V
1 1 V
2 1 V
1 2 V
1 1 V
2 2 V
1 1 V
2 1 V
1 2 V
1 1 V
2 1 V
1 2 V
2 1 V
1 1 V
1 2 V
2 1 V
1 1 V
2 2 V
1 1 V
2 2 V
1 1 V
1 1 V
2 2 V
1 1 V
2 1 V
1 2 V
1 1 V
2 1 V
1 2 V
2 1 V
1 1 V
1 2 V
2 1 V
1 2 V
2 1 V
1 1 V
2 2 V
1 1 V
1 1 V
2 2 V
1 1 V
2 1 V
1 2 V
1 1 V
2 2 V
1 1 V
2 1 V
1 2 V
1 1 V
2 1 V
1 2 V
2 1 V
1 1 V
1 2 V
2 1 V
1 2 V
2 1 V
1 1 V
stroke 6099 1514 M
2 2 V
1 1 V
1 1 V
2 2 V
1 1 V
2 2 V
1 1 V
1 1 V
2 2 V
1 1 V
2 1 V
1 2 V
1 1 V
2 2 V
1 1 V
2 1 V
1 2 V
1 1 V
2 1 V
1 2 V
2 1 V
1 1 V
2 2 V
1 1 V
1 2 V
2 1 V
1 1 V
2 2 V
1 1 V
1 1 V
2 2 V
1 1 V
2 2 V
1 1 V
1 1 V
2 2 V
1 1 V
2 2 V
1 1 V
2 1 V
1 2 V
1 1 V
2 1 V
1 2 V
2 1 V
1 2 V
1 1 V
2 1 V
1 2 V
2 1 V
1 1 V
1 2 V
2 1 V
1 2 V
2 1 V
1 1 V
1 2 V
2 1 V
1 1 V
2 2 V
1 1 V
2 2 V
1 1 V
1 1 V
2 2 V
1 1 V
2 2 V
1 1 V
1 1 V
2 2 V
1 1 V
2 1 V
1 2 V
1 1 V
2 2 V
1 1 V
2 1 V
1 2 V
2 1 V
1 2 V
1 1 V
2 1 V
1 2 V
2 1 V
1 1 V
1 2 V
2 1 V
1 2 V
2 1 V
1 1 V
1 2 V
2 1 V
1 2 V
2 1 V
1 1 V
1 2 V
2 1 V
1 1 V
2 2 V
1 1 V
2 2 V
1 1 V
1 1 V
2 2 V
stroke 6246 1658 M
1 1 V
2 2 V
1 1 V
1 1 V
2 2 V
1 1 V
2 2 V
1 1 V
1 1 V
2 2 V
1 1 V
2 1 V
1 2 V
2 1 V
1 2 V
1 1 V
2 1 V
1 2 V
2 1 V
1 2 V
1 1 V
2 1 V
1 2 V
2 1 V
1 2 V
1 1 V
2 1 V
1 2 V
2 1 V
1 1 V
1 2 V
2 1 V
1 2 V
2 1 V
1 1 V
2 2 V
1 1 V
1 2 V
2 1 V
1 1 V
2 2 V
1 1 V
1 2 V
2 1 V
1 1 V
2 2 V
1 1 V
1 2 V
2 1 V
1 1 V
2 2 V
1 1 V
1 1 V
2 2 V
1 1 V
2 2 V
1 1 V
2 1 V
1 2 V
1 1 V
2 2 V
1 1 V
2 1 V
1 2 V
1 1 V
2 2 V
1 1 V
2 1 V
1 2 V
1 1 V
2 2 V
1 1 V
2 1 V
1 2 V
2 1 V
1 2 V
1 1 V
2 1 V
1 2 V
2 1 V
1 1 V
1 2 V
2 1 V
1 2 V
2 1 V
1 1 V
1 2 V
2 1 V
1 2 V
2 1 V
1 1 V
1 2 V
2 1 V
1 2 V
2 1 V
1 1 V
2 2 V
1 1 V
1 2 V
2 1 V
1 1 V
2 2 V
1 1 V
1 1 V
stroke 6392 1802 M
2 2 V
1 1 V
2 2 V
1 1 V
1 1 V
2 2 V
1 1 V
2 2 V
1 1 V
2 1 V
1 2 V
1 1 V
2 2 V
1 1 V
2 1 V
1 2 V
1 1 V
2 2 V
1 1 V
2 1 V
1 2 V
1 1 V
2 1 V
1 2 V
2 1 V
1 2 V
1 1 V
2 1 V
1 2 V
2 1 V
1 2 V
2 1 V
1 1 V
1 2 V
2 1 V
1 2 V
2 1 V
1 1 V
1 2 V
2 1 V
1 2 V
2 1 V
1 1 V
1 2 V
2 1 V
1 1 V
2 2 V
1 1 V
1 2 V
2 1 V
1 1 V
2 2 V
1 1 V
2 2 V
1 1 V
1 1 V
2 2 V
1 1 V
2 2 V
1 1 V
1 1 V
2 2 V
1 1 V
2 1 V
1 2 V
1 1 V
2 2 V
1 1 V
2 1 V
1 2 V
2 1 V
1 2 V
1 1 V
2 1 V
1 2 V
2 1 V
1 1 V
1 2 V
2 1 V
1 2 V
2 1 V
1 1 V
1 2 V
2 1 V
1 2 V
2 1 V
1 1 V
1 2 V
2 1 V
1 1 V
2 2 V
1 1 V
2 2 V
1 1 V
1 1 V
2 2 V
1 1 V
2 2 V
1 1 V
1 1 V
2 2 V
1 1 V
2 1 V
1 2 V
stroke 6539 1947 M
1 1 V
2 2 V
1 1 V
2 1 V
1 2 V
2 1 V
1 2 V
1 1 V
2 1 V
1 2 V
2 1 V
1 1 V
1 2 V
2 1 V
1 2 V
2 1 V
1 1 V
1 2 V
2 1 V
1 1 V
2 2 V
1 1 V
1 2 V
2 1 V
1 1 V
2 2 V
1 1 V
2 1 V
1 2 V
1 1 V
2 2 V
1 1 V
2 1 V
1 2 V
1 1 V
2 1 V
1 2 V
2 1 V
1 2 V
1 1 V
2 1 V
1 2 V
2 1 V
1 1 V
1 2 V
2 1 V
1 2 V
2 1 V
1 1 V
2 2 V
1 1 V
1 1 V
2 2 V
1 1 V
2 2 V
1 1 V
1 1 V
2 2 V
1 1 V
2 1 V
1 2 V
1 1 V
2 1 V
1 2 V
2 1 V
1 2 V
2 1 V
1 1 V
1 2 V
2 1 V
1 1 V
2 2 V
1 1 V
1 2 V
2 1 V
1 1 V
2 2 V
1 1 V
1 1 V
2 2 V
1 1 V
2 1 V
1 2 V
1 1 V
2 2 V
1 1 V
2 1 V
1 2 V
2 1 V
1 1 V
1 2 V
2 1 V
1 1 V
2 2 V
1 1 V
1 1 V
2 2 V
1 1 V
2 2 V
1 1 V
1 1 V
2 2 V
1 1 V
2 1 V
stroke 6686 2089 M
1 2 V
1 1 V
2 1 V
1 2 V
2 1 V
1 2 V
2 1 V
1 1 V
1 2 V
2 1 V
1 1 V
2 2 V
1 1 V
1 1 V
2 2 V
1 1 V
2 1 V
1 2 V
1 1 V
2 1 V
1 2 V
2 1 V
1 1 V
2 2 V
1 1 V
1 2 V
2 1 V
1 1 V
2 2 V
1 1 V
1 1 V
2 2 V
1 1 V
2 1 V
1 2 V
1 1 V
2 1 V
1 2 V
2 1 V
1 1 V
1 2 V
2 1 V
1 1 V
2 2 V
1 1 V
2 1 V
1 2 V
1 1 V
2 1 V
1 2 V
2 1 V
1 1 V
1 2 V
2 1 V
1 2 V
2 1 V
1 1 V
1 2 V
2 1 V
1 1 V
2 2 V
1 1 V
2 1 V
1 2 V
1 1 V
2 1 V
1 2 V
2 1 V
1 1 V
1 2 V
2 1 V
1 1 V
2 2 V
1 1 V
1 1 V
2 2 V
1 1 V
2 1 V
1 2 V
1 1 V
2 1 V
1 2 V
2 1 V
1 1 V
2 2 V
1 1 V
1 1 V
2 1 V
1 2 V
2 1 V
1 1 V
1 2 V
2 1 V
1 1 V
2 2 V
1 1 V
1 1 V
2 2 V
1 1 V
2 1 V
1 2 V
1 1 V
2 1 V
1 2 V
stroke 6832 2229 M
2 1 V
1 1 V
2 2 V
1 1 V
1 1 V
2 2 V
1 1 V
2 1 V
1 2 V
1 1 V
2 1 V
1 1 V
2 2 V
1 1 V
1 1 V
2 2 V
1 1 V
2 1 V
1 2 V
2 1 V
1 1 V
1 2 V
2 1 V
1 1 V
2 1 V
1 2 V
1 1 V
2 1 V
1 2 V
2 1 V
1 1 V
1 2 V
2 1 V
1 1 V
2 2 V
1 1 V
1 1 V
2 1 V
1 2 V
2 1 V
1 1 V
2 2 V
1 1 V
1 1 V
2 2 V
1 1 V
2 1 V
1 1 V
stroke
LTb
1260 4800 N
0 -4400 V
5640 0 V
0 4400 V
-5640 0 V
Z stroke
1.000 UP
0.500 UL
LTb
grestore % colour palette end
stroke
grestore
end
showpage
  }}%
  \put(6900,200){\makebox(0,0){\strut{}11}}%
  \put(6195,200){\makebox(0,0){\strut{}10.5}}%
  \put(5490,200){\makebox(0,0){\strut{}10}}%
  \put(4785,200){\makebox(0,0){\strut{}9.5}}%
  \put(4080,200){\makebox(0,0){\strut{}9}}%
  \put(3375,200){\makebox(0,0){\strut{}8.5}}%
  \put(2670,200){\makebox(0,0){\strut{}8}}%
  \put(1965,200){\makebox(0,0){\strut{}7.5}}%
  \put(1260,200){\makebox(0,0){\strut{}7}}%
  \put(1140,4643){\makebox(0,0)[r]{\strut{}0.015}}%
  \put(1140,3857){\makebox(0,0)[r]{\strut{}0.01}}%
  \put(1140,3071){\makebox(0,0)[r]{\strut{}0.005}}%
  \put(1140,2286){\makebox(0,0)[r]{\strut{}0}}%
  \put(1140,1500){\makebox(0,0)[r]{\strut{}-0.005}}%
  \put(1140,714){\makebox(0,0)[r]{\strut{}-0.01}}%
\end{picture}%
\endgroup
\endinput

  \end{center}
\end{figure}

$$f(x) = \frac{\cos x}{x^2}
,$$
$$f'(x) = -\frac{\sin x}{x^2} - 2\frac{\cos x}{x^3}
,$$
$$f''(x) = ../../input/function_der2.tex.$$

Из графика видно, что исходная функция унимодальна на исследуемом промежутке.

\section{Описание алгоритма}
Общая идея поиска минимума унимодальной функции на отрезке заключается в том, 
что отрезок с минимумом можно разбить на части и гарантированно определить в какой части находится минимум,
тем самым позволив сузить область для поиска минимума.

В случае метода золотого сечения, отрезок $[a,b]$, на котором имеется минимум, делится в отношении 
\textit{золотой пропорции}.
\textit{Золотая пропорция}~--- это деление непрерывной величины на части в таком отношении, 
что большая часть относится к меньшей, как большая ко всей величине.

$x \in [a,b]$ делит $[a,b]$ золотым сечением, если 
$$
\begin{cases}
  x - a > b - x &  \frac{x - a}{b - x} = \frac{b - a}{x - a} = \varphi \\
  x - a < b - x &  \frac{b - x}{x - a} = \frac{b - a}{b - x} = \varphi \\
\end{cases},
$$
т.\,е.~для любого непустого отрезка найдутся две точки, делящие его в отношении золотой пропорции.

Величина отношения $\varphi$ определена единственным образом, и равна $\frac{\sqrt{5} + 1}{2}.$

Поиск минимума унимодальной функции методом золотого сечения можно описать алгоритмом~\ref{FindMinimumByGoldenSectionSearch}:
\begin{algorithm}[H]
\caption{FindMinimumByGoldenSectionSearch. Поиск минимума унимодальной функции.}
\label{FindMinimumByGoldenSectionSearch}
\INPUT Исследуемая функция $f$; отрезок $[a,b]$, содержащий минимум; точность $\varepsilon$, с которой требуется найти минимум. \\
\OUTPUT Точка $x^*$~--- аргумент функции в которой достигается минимум с точностью $\varepsilon$.

\begin{algorithmic}
\STATE \COMMENT { Инициализируем первоначальное деление отрезка $[a,b]$ на три отрезка золотым сечением. }
\STATE $\alpha$ := $\frac{1}{\varphi}$ \COMMENT { с данной величиной далее будет удобно работать }
\STATE $i$ := 0 \COMMENT { счетчик итераций }
\STATE \COMMENT { $[a_i,b_i]$~--- отрезок содержащий минимум на $i$-м шаге. }
\STATE $a_i$ := $a$
\STATE $b_i$ := $b$
\STATE \COMMENT { $\lambda_i < \mu_i$~--- точки делящие отрезок на $i$-м шаге золотым сечением. }
\STATE $\lambda_i$ := $a_i + (1 - \alpha)(b_i - a_i)$
\STATE $\mu_i$ := $a_i + \alpha(b_i - a_i)$
\STATE \COMMENT { Подразбиваем отрезок пока его длина не станет меньше необходимой точности. }
\WHILE { $|a_i - b_i| \geqslant \varepsilon$ }
  \IF { $f(\lambda_i) \leqslant f(\mu_i)$ }
    \STATE \COMMENT { Минимум лежит на $[a_i,\mu_i]$, подразбиваем этот отрезок и продолжаем работу алгоритма. }
    \STATE $a_{i+1}$ := $a_i$
    \STATE $b_{i+1}$ := $\mu_i$
    \STATE \COMMENT { Из-за особенностей золотого сечения необходимо перевычислять лишь одну новую точку. }
    \STATE $\mu_{i+1}$ := $\lambda_i$
    \STATE $\lambda_{i+1}$ := $a_{i+1} + (1 - \alpha)(b_{i+1} - a_{i+1})$
  \ELSE
    \STATE \COMMENT { Минимум лежит на $[\lambda_i,b_i]$. }
    \STATE $a_{i+1}$ := $\lambda_i$
    \STATE $b_{i+1}$ := $b_i$
    \STATE $\lambda_{i+1}$ := $\mu_i$
    \STATE $\mu_{i+1}$ := $a_{i+1} + \alpha(b_{i+1} - a_{i+1})$
  \ENDIF
  \STATE $i$ := $i + 1$
\ENDWHILE
\STATE \COMMENT { Все точки на результирующем отрезке равны точке минимума с точностью $\varepsilon$, %
  выбираем точку с меньшим значением функции из $\lambda_i$ и $\mu_i$.  }
\IF { $f(\lambda_i) \leqslant f(\mu_i)$ }
  \RETURN $\lambda_i$
\ELSE
  \RETURN $\mu_i$
\ENDIF
\end{algorithmic}
\end{algorithm}

Приведённый алгоритм~\ref{FindMinimumByGoldenSectionSearch} может быть легко оптимизирован, 
если сохранять вычисленные значения функции $f$ в точках $\alpha_i$ и $\mu_i$, 
таким образом функция может быть вычислена лишь минимально необходимое число раз,
для простоты эта оптимизация не приводится в алгоритме.

\section{Код программы}
\lstset{language=Octave, caption=Решение задачи поиска минимума унимодальной функции,%
label=main-source-code, basicstyle=\footnotesize,%
numbers=left, numberstyle=\footnotesize, numbersep=5pt, frame=single, breaklines=true, breakatwhitespace=false,%
inputencoding=utf8x}
\lstinputlisting{data/golden_section_search.m}

\section{Результаты решения}
Результаты решения приведены в таблице \ref{result-table}.

\begin{table}[H]
\caption{Результаты работы метода золотого сечения}
\label{result-table}
\begin{center}
\begin{tabular}{|c|c|c|c|c|c|c|c|}
\hline
{\small Точность} & {\small Итер.} & {\small Вызовы $f$} & $x$ & $f(x)$ & $f(x_i) - f(x_{i-1})$ & $f'(x)$ & $f''(x)$ \\
\hline
%TODO: Formatting.
1.0e-03 & 18 & 20 & 9.210999446 & -0.011518238 &  & 4.133279241e-07 \\
1.0e-04 & 23 & 25 & 9.210960866 & -0.011518238 & -7.172261e-12 & -4.151203903e-08 \\
1.0e-05 & 27 & 29 & 9.210964345 & -0.011518238 & -7.307176e-14 & -4.988289602e-10 \\
1.0e-06 & 32 & 34 & 9.210964345 & -0.011518238 & 0.000000e+00 & -4.988289602e-10 \\
1.0e-07 & 37 & 39 & 9.210964391 & -0.011518238 & -1.214306e-17 & 4.072474880e-11 \\
1.0e-08 & 42 & 44 & 9.210964391 & -0.011518238 & 0.000000e+00 & 4.072474880e-11 \\

\hline
\end{tabular}
\end{center}
\end{table}


\section{Возможные дополнительные исследования}
Вследствии особенностей золотого сечения, на каждой итерации длина исследуемого промежутка уменьшается в $\varphi$ раз, 
что позволяет заранее оценить необходимое число итераций для нахождения минимума с определённой точностью:
$$
  l_i = l_0 \frac{1}{\varphi^i} = \varepsilon,
$$
$$
   i = \log_{\varphi} \frac{l_0}{\varepsilon}.
$$

На каждой итерации функция вычисляется ровно один раз.

\section{Обоснование достоверности полученного результата}
Согласно графику, исследуемая функция выпукла вверх вблизи локального минимума, 
$f''(x^*) > 0,$
и полученный результат с допустимой точностью обращает производную исследуемой функции в ноль, 
значит найденная точка является точкой локального минимума.

\end{document}
