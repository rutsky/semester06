% 2d_minimisation.tex
% Report on 2d minimisations algorithms (gradient descent and genetic algorithm).
% Vladimir Rutsky <altsysrq@gmail.com>
% 28.03.2009

\documentclass[10pt,a4paper,titlepage]{article}

% Spaces after commas.
\frenchspacing
% Minimal carrying number of characters,
\righthyphenmin=2

% Encoding support.
\usepackage{ucs}
\usepackage[utf8x]{inputenc}
\usepackage[T2A]{fontenc}
\usepackage[russian]{babel}

\usepackage{amsmath, amsthm, amssymb}

% From K.V.Voroncov Latex in samples, 2005.
\textheight=24cm   % Text height.
\textwidth=16cm    % Text width
\oddsidemargin=0pt % Left side indention.
\topmargin=-1.5cm  % Top side indention.
\parindent=24pt    % Paragraph indent.
\parskip=0pt       % Distance between paragraphs.
\tolerance=2000
%\flushbottom       % Page height aligning.

% Algorithms stuff.
\usepackage{algorithm}
\usepackage{algorithmic}

\newcommand{\INPUT}{\textbf{Вход:~}}
\newcommand{\OUTPUT}{\textbf{Выход:~}}
\floatname{algorithm}{Алгоритм}
\renewcommand{\algorithmicrepeat}{\textbf{do}}
\renewcommand{\algorithmicuntil}{\textbf{while}}

% Drawing figures.
\usepackage{subfig}

% Drawing graphs.
\usepackage{tikz}
\usetikzlibrary{trees,positioning,arrows}

% For listings.
\usepackage{listings}

% Transpose symbol in math mode.
\newcommand\T{^\mathrm{\textsf{\tiny{T}}}}
% Real number space.
\newcommand\RR{\mathbb{R}}

% Source code listings.
\renewcommand{\lstlistingname}{Исходный код}


\title{Решение задачи многомерной минимизации функции}
\author{Владимир Руцкий, 3057/2}
%\date{}

\begin{document}

\maketitle
\thispagestyle{empty}

% Content.
\section{Постановка задачи}
Требуется найти с наперёд заданной точностью минимум многомерной функции $f(x)$ в некоторой области:
$$ \min f(x), \quad x \in \RR^n, $$
используя \textit{метод градиентного спуска} и \textit{генетический алгоритм}.

Исходная функция: $f(x) = \frac{\cos x}{x^2}
,$ заданная на $\RR^2$. %TODO: Dimension specified directly.

\section{Исследование применимости методов}
\subsection{Метод градиентного спуска}
\label{GradientDescentMethodConditions}
\textit{Метод градиентного спуска} основывается на том, что для гладкой выпуклой функции градиент функции в точке направлен
в сторону увеличения функции.
Используя этот факт можно построить итерационный процесс. 
Выберем начальное приближение минимума, далее построим последовательность точек,
в которой каждая следующая точка выбирается на луче противоположном градиенту в текущей точке:
$$ x_{k+1} = x_k - \lambda_k \nabla f(x_k). $$
Шаг на который двигается текущая точка за одну итерацию равен $\lambda_k$ и может задаваться различными способами, например:
\begin{enumerate}
  \item $\lambda_k = \mathrm{const}$, фиксированный шаг;
  \item $\lambda_k = c \lambda_{k-1}, \quad 0 < c < 1,$ равномерно уменьшающийся шаг;
  \item $\lambda_k \in (\,0, q\,)\!: 
    \quad f(x_k - \lambda_k \nabla f(x_k)) = \min\limits_{0 < \lambda < q} f(x_k - \lambda \nabla f(x_k)),$
    в качестве следующей точки выбирается точка в которой достигается минимум на отрезке уменьшения функции,
    направленным против градиента.
\end{enumerate}

Для того, чтобы к функции можно было применить указанный выше итерационный процесс, необходимо, 
чтобы функция была гладкой: $f \in C^{1,1}$.

%FIXME: Условие сходимости.

Критерием остановки итерационного является событие, когда следующая точка находится от предыдущей 
на расстоянии меньше либо равном $\varepsilon$:
$$ || x_{k+1} - x_k || < \varepsilon. $$
% TODO: Критерий остановки по малой величине градиента.

Исходная функция в исследуемой области % FIXME!
удовлетворяет необходимым для сходимости метода градиентного спуска условиям, указанным в \ref{GradientDescentMethodConditions}.

\subsection{Генетический алгоритм}

%FIXME:

\section{Описание алгоритма}
\subsection{Метод градиентного спуска}

В используемой реализации алгоритма $\lambda_k$ выбирается таким согласно последнему методу, указанному в \ref{GradientDescentMethodConditions}:
$$
  \lambda_k \in (\,0, q\,)\!: 
    \quad f(x_k - \lambda_k \nabla f(x_k)) = \min\limits_{0 < \lambda < q} f(x_k - \lambda \nabla f(x_k)),
$$
значение $\lambda_k$ ищется методом золотого сечения.

%FIXME: Алгоритм.

\subsection{Генетический алгоритм}
%FIXME:

\section{Код программы}
\subsection{Метод градиентного спуска}
\lstset{language=C++, caption=Градиентный спуск,%
label=gd-source-code, basicstyle=\footnotesize,%
numbers=left, numberstyle=\footnotesize, numbersep=5pt, frame=single, breaklines=true, breakatwhitespace=false,%
inputencoding=utf8x}
\lstinputlisting{data/gradient_descent.hpp}

\subsection{Генетический алгоритм}
%FIXME:

\section{Результаты решения}
\subsection{Метод градиентного спуска}
Результаты решения приведены в таблице \ref{gd-result-table}. 

%TODO: spaces!
Начальной точкой была выбрана точка FIXME, 
шаг для поиска минимума методом золотого сечения был равен ../../input/gd_step.tex.

\begin{table}[H]
\caption{Результаты работы алгоритма градиентного спуска}
\label{gd-result-table}
\begin{center}
\begin{tabular}{|c|c|c|c|}
\hline
Точность & Количеcтво шагов & $x$ & $f(x)$ \\
\hline
%TODO: Formatting.
1e-03 & 12 & (4.61855e-06,     -0.81648024) &      4.89897949 \\
1e-04 & 12 & (-8.49295e-07,     -0.81646860) &      4.89897949 \\
1e-05 & 13 & (1.20244e-07,     -0.81649645) &      4.89897949 \\
1e-06 & 13 & (4.11864e-07,     -0.81649654) &      4.89897949 \\
1e-07 & 14 & (1.41643e-08,     -0.81649657) &      4.89897949 \\
1e-08 & 17 & (3.07062e-09,     -0.81649657) &      4.89897949 \\

\hline
\end{tabular}
\end{center}
\end{table}

\subsection{Генетический алгоритм}
%FIXME:

%\section{Возможные дополнительные исследования}
%\subsection{Метод градиентного спуска}
%FIXME:

%\subsection{Генетический алгоритм}
%FIXME:

\section{Обоснование достоверности полученного результата}
%\subsection{Метод градиентного спуска}
%FIXME:

\subsection{Генетический алгоритм}
%FIXME:

\end{document}
