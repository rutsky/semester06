% 2d_minimisation_newton.tex
% Report on 2d minimisations algorithms (Newton method).
% Vladimir Rutsky <altsysrq@gmail.com>
% 21.04.2009

\documentclass[10pt,a4paper,titlepage]{article}

% Spaces after commas.
\frenchspacing
% Minimal carrying number of characters,
\righthyphenmin=2

% Encoding support.
\usepackage{ucs}
\usepackage[utf8x]{inputenc}
\usepackage[T2A]{fontenc}
\usepackage[russian]{babel}

\usepackage{amsmath, amsthm, amssymb}

% From K.V.Voroncov Latex in samples, 2005.
\textheight=24cm   % Text height.
\textwidth=16cm    % Text width
\oddsidemargin=0pt % Left side indention.
\topmargin=-1.5cm  % Top side indention.
\parindent=24pt    % Paragraph indent.
\parskip=0pt       % Distance between paragraphs.
\tolerance=2000
%\flushbottom       % Page height aligning.

% For listings.
\usepackage{listings}

% Альбомная ориентация страниц % TODO: Comment.
\usepackage{lscape}

% Transpose symbol in math mode.
\newcommand\T{^\mathrm{\textsf{\tiny{T}}}}
% Real number space.
\newcommand\RR{\mathbb{R}}

% Source code listings.
\renewcommand{\lstlistingname}{Исходный код}


\begin{document}

% Source code.
\lstset{language=C++, caption=Метод Ньютона,%
label=nm-source-code, basicstyle=\footnotesize,%
numbers=left, numberstyle=\footnotesize, numbersep=5pt, frame=single, breaklines=true, breakatwhitespace=false,%
inputencoding=utf8x}
\lstinputlisting{data/newton.hpp}


% Results.
%\pagebreak
%
%\begin{landscape}
%
%\thispagestyle{empty}
%
%\begin{table}[H]
%\caption{Метод Ньютона}
%\label{nm-result-table}
%\begin{center}
%\begin{tabular}{|c|c|c|c|c|c|}
%\hline
%Точность & Шаги & $x$ & $f(x)$ & $f_i(x) - f_{i - 1}(x)$ & $\nabla f(x)$\\
%\hline
%TODO: Formatting.
%1e-03 & 11 & (     0.00001671,     -0.81627145) &      4.89897953 &  & (4.093416e-05, 4.136302e-04) \\
1e-04 & 12 & (    -0.00000069,     -0.81649038) &      4.89897949 & -4.686444e-08 & (-1.694262e-06, 1.139126e-05) \\
1e-05 & 12 & (    -0.00000003,     -0.81649574) &      4.89897949 & -3.525447e-11 & (-8.537893e-08, 1.541512e-06) \\
1e-06 & 12 & (     0.00000000,     -0.81649664) &      4.89897949 & -6.457057e-13 & (6.699014e-09, -1.043777e-07) \\
1e-07 & 12 & (    -0.00000000,     -0.81649657) &      4.89897949 & -3.552714e-15 & (-5.828198e-10, 2.497871e-08) \\
1e-08 & 15 & (     0.00000000,     -0.81649659) &      4.89897949 & 0.000000e+00 & (8.046178e-10, -2.021831e-08) \\

%\hline
%\end{tabular}
%\end{center}
%\end{table}
%
%\end{landscape}

\end{document}
