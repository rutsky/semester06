% barrier_method.tex
% Som kind of report data on barrier method algorithm.
% Vladimir Rutsky <altsysrq@gmail.com>
% 05.05.2009

\documentclass[10pt,a4paper,titlepage]{article}

% Spaces after commas.
\frenchspacing
% Minimal carrying number of characters,
\righthyphenmin=2

% Encoding support.
\usepackage{ucs}
\usepackage[utf8x]{inputenc}
\usepackage[T2A]{fontenc}
\usepackage[russian]{babel}

\usepackage{amsmath, amsthm, amssymb}

% From K.V.Voroncov Latex in samples, 2005.
\textheight=24cm   % Text height.
\textwidth=16cm    % Text width
\oddsidemargin=0pt % Left side indention.
\topmargin=-1.5cm  % Top side indention.
\parindent=24pt    % Paragraph indent.
\parskip=0pt       % Distance between paragraphs.
\tolerance=2000
%\flushbottom       % Page height aligning.

% Algorithms stuff.
\usepackage{algorithm}
\usepackage{algorithmic}

\newcommand{\INPUT}{\textbf{Вход:~}}
\newcommand{\OUTPUT}{\textbf{Выход:~}}
\floatname{algorithm}{Алгоритм}
\renewcommand{\algorithmicrepeat}{\textbf{do}}
\renewcommand{\algorithmicuntil}{\textbf{while}}

% Images.
%\usepackage[pdftex]{graphicx}

% Drawing figures.
\usepackage{subfig}

% Drawing graphs.
\usepackage{tikz}
\usetikzlibrary{trees,positioning,arrows}

% For listings.
\usepackage{listings}

% Альбомная ориентация страниц % TODO: Comment.
\usepackage{lscape}


% Transpose symbol in math mode.
\newcommand\T{^\mathrm{\textsf{\tiny{T}}}}
% Real number space.
\newcommand\RR{\mathbb{R}}

% Source code listings.
\renewcommand{\lstlistingname}{Исходный код}


\title{Решение задачи многомерной минимизации функции с~ограничениями методом барьеров}
\author{Владимир Руцкий, 3057/2}
\date{} % cheat ;)

\begin{document}

\maketitle
\thispagestyle{empty}

% Content.
%\section{Постановка задачи}
%\section{Исследование применимости метода}
%\section{Описание алгоритма}
\section*{Код программы}
\lstset{language=C++, caption=Барьерный метод,%
label=bm-source-code, basicstyle=\footnotesize,%
numbers=left, numberstyle=\footnotesize, numbersep=5pt, frame=single, breaklines=true, breakatwhitespace=false,%
inputencoding=utf8x}
\lstinputlisting{data/barrier_method.hpp}

\section*{Результаты решения}
%Результаты решения приведены в таблице \ref{gd-result-table}.

%TODO: spaces!
%Начальной точкой была выбрана точка FIXME\!, % TODO: \!
%шаг для поиска минимума методом золотого сечения был равен ../../input/gd_step.tex\!. % TODO: \!
% TODO: Рисунок с шагами.

\begin{landscape}
\begin{table}[H]
\caption{Результаты работы барьерного метода}
\label{bm-result-table}
\begin{center}
\begin{tabular}{|c|c|c|c|c|c|c|c|}
\hline
Точность & Шаги & $x$ & $f(x)$ & $f_i(x) - f_{i - 1}(x)$ & $\nabla f(x)$ & $g_1(x)$ &  $g_2(x)$ \\
\hline
%TODO: Formatting.
1e-01 & 8 & (0.657642,      0.63160031) &    -10.79780664 &  & (-8.68472, -6.736799e+00) \\
1e-02 & 9 & (0.657642,      0.63160031) &    -10.79780664 & 0.000000e+00 & (-8.68472, -6.736799e+00) \\
1e-03 & 10 & (0.657642,      0.63160031) &    -10.79780664 & 0.000000e+00 & (-8.68472, -6.736799e+00) \\
 % TODO: Correct file name.
\hline
\end{tabular}
\end{center}
\end{table}
\end{landscape}

%\section{Возможные дополнительные исследования}
%\section{Обоснование достоверности полученного результата}

\end{document}
