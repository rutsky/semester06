% linear_programming_problem.tex
% Report on solving linear programming problems.
% Vladimir Rutsky <altsysrq@gmail.com>
% 23.03.2009

\documentclass[10pt,a4paper,titlepage]{article}

% Spaces after commas.
\frenchspacing
% Minimal carrying number of characters,
\righthyphenmin=2

% Encoding support.
\usepackage{ucs}
\usepackage[utf8x]{inputenc}
\usepackage[T2A]{fontenc}
\usepackage[russian]{babel}

\usepackage{amsmath, amsthm, amssymb}

% From K.V.Voroncov Latex in samples, 2005.
\textheight=24cm   % Text height.
\textwidth=16cm    % Text width
\oddsidemargin=0pt % Left side indention.
\topmargin=-1.5cm  % Top side indention.
\parindent=24pt    % Paragraph indent.
\parskip=0pt       % Distance between paragraphs.
\tolerance=2000
%\flushbottom       % Page height aligning.

% Algorithms stuff.
\usepackage{algorithm}
\usepackage{algorithmic}

\newcommand{\INPUT}{\textbf{Вход:~}}
\newcommand{\OUTPUT}{\textbf{Выход:~}}
\floatname{algorithm}{Алгоритм}
\renewcommand{\algorithmicrepeat}{\textbf{do}}
\renewcommand{\algorithmicuntil}{\textbf{while}}

% Drawing figures.
\usepackage{subfig}

% Drawing graphs.
\usepackage{tikz}
\usetikzlibrary{trees,positioning,arrows}

% For listings.
\usepackage{listings}

% Transpose symbol in math mode.
\newcommand\T{^\mathrm{\textsf{\tiny{T}}}}
% Real number space.
\newcommand\RR{\mathbb{R}}

% Source code listings.
\renewcommand{\lstlistingname}{Исходный код}


\title{Решение задачи линейного программирования}
\author{Владимир Руцкий, 3057/2}
%\date{}

\begin{document}

\maketitle
\thispagestyle{empty}

% Content.
\section{Постановка задачи}
Целью данной работы является изучение нескольких методов решения \textit{задачи линейного программирования}, 
которая без ограничения общности может быть представлена в \textit{канонической форме}:
$$ \min c\T x, \quad \forall x \in S $$
$$ S = \{ \, x\ |\ a_i\T x - b^{(i)} = 0, \quad i = \overline{1,\!M}, x \geqslant 0 \, \},$$
где $c\T x$~--- целевая функция, а множество $S$~--- множество допустимых точек.

Результатом решения задачи линейного программирования в представленной выше форме является один из следующих выводов:
\begin{enumerate}
  \item Множество допустимых точек непусто и найдётся точка $x_* \in S$ такая, что $c\T x_* \leqslant c\T x, \quad \forall x \in S$. 
Такая точка $x_*$ называется \textit{оптимальной}.
  \item Множество допустимых точек непусто и функция $c\T x$ неограничена снизу при $x \in S$.
  \item Множество допустимых точек пусто: $S \ne \varnothing$.
\end{enumerate}

Каноническую задачу линейного программирования можно представить в матричной форме:
$$ \min c\T[N] \, x[N], \quad x[N] \in S $$
$$ S = \{ \, x[N]\ |\ A[M,N] \, x[N] = b[M], \quad x[N] \geqslant 0 \, \}.$$

В данной работе решается задача для следующих исходных данных:
$$A[M,N] = \left( \begin{array}{cccccc}
-1 & 3 & 0 & 1 & 0 & 0 \\
4 & 1 & 0 & 0 & 1 & 0 \\
-2 & 1 & 1 & 0 & 0 & 0 \\
1 & -3 & 0 & 0 & 0 & 1 \\
\end{array}
 \right), \quad
b[M] = \left( \begin{array}{c}
13 \\
26 \\
1 \\
0 \\
\end{array}
 \right)$$
$$c\T[N] = \left( \begin{array}{cccccc}
4 & -3 & 0 & -1 & 1 & 0 
\end{array}
 \right)$$

\section{Исследование применимости методов}

\subsection{Симплекс метод} % + двойственная задача
Симплекс метод применим для решения любой задачи линейного программирования.

Однако используемая конкретная реализация симплекс метода требует, 
чтобы задача была приведена к каноническому виду и 
ранг матрицы ограничений $A$ был полным, 
что выполняется в данных условиях.

\subsection{Метод перебора крайних точек множества допустимых точек}
\subsection{Генетический алгоритм}

\section{Описание алгоритма}

\subsection{Симплекс метод} % + двойственная задача
\subsection{Метод перебора крайних точек множества допустимых точек}
\subsection{Генетический алгоритм}

\section{Код программы}

\subsection{Симплекс метод} % + двойственная задача
\subsection{Метод перебора крайних точек множества допустимых точек}
\subsection{Генетический алгоритм}

\section{Результаты решения}

\subsection{Симплекс метод} % + двойственная задача
\subsection{Метод перебора крайних точек множества допустимых точек}
\subsection{Генетический алгоритм}

\section{Возможные дополнительные исследования}
\section{Обоснование достоверности полученного результата}

\end{document}
