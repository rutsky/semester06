% ed_system_reform.tex
% Term paper on Economics.
% Vladimir Rutsky <altsysrq@gmail.com>
% 23.05.2009

\documentclass[10pt,a4paper,titlepage]{article}

% Spaces after commas.
\frenchspacing
% Minimal carrying number of characters,
\righthyphenmin=2

% Encoding support.
\usepackage{ucs}
\usepackage[utf8x]{inputenc}
\usepackage[T2A]{fontenc}
\usepackage[russian]{babel}
% URL.
\usepackage{url}
% Images.
\usepackage[dvips]{graphicx}
\graphicspath{{./}}

% From K.V.Voroncov Latex in samples, 2005.
\textheight=24cm   % Text height.
\textwidth=16cm    % Text width
\oddsidemargin=0pt % Left side indention.
\topmargin=-1.5cm  % Top side indention.
\parindent=24pt    % Paragraph indent.
\parskip=0pt       % Distance between paragraphs.
\tolerance=2000
%\flushbottom       % Page height aligning.

\title{Реформы системы образования в России}
\author{Владимир Руцкий, 3057/2}

\begin{document}

\titlepage
\maketitle

% Content.
\section*{Цель работы}
Целью данной работы является анализ текущей системы образования с финансовой стороны, 
насколько вложенные в обучение студентов деньги возвращаются обратно государству, 
и как можно было бы реформировать систему образования для увеличения ``прибыльности'' 
системы образования.

\section*{Анализ проблемы}
На текущий момент в России существует возможность бесплатного обучения на всех уровнях образования 
(школа, ВУЗ, аспирантура).
При этом далеко не всегда человек, получивший образование, 
идёт работать по специальности или вообще остаётся работать в нашей стране, 
и получается, что либо предоставленное бесплатное образование вообще не было нужно человеку, 
либо будет использовано им при работе на другое государство.
Это так называемая ``утечка мозгов''. %TODO: привести статистику

\section*{Возможные решения проблемы}
Как можно обеспечить возврат денег государству, вложенных в обучение человека?

\paragraph{Платное обучение} 
Можно сделать всё обучение, кроме самого базового, платным. 
При этом государство гарантировано не потеряет деньги от вложений в систему образования, 
т.к. практически ничего вкладывать и не будет.
Так сделано, например, в США и Китае.

Проблема этого подхода состоит в том, что число людей, получивших образование, существенно снизится, 
из-за весьма низкого дохода населения.
Произойдёт разделение общества на элиту, способную оплачивать обучение,
и большинства остальных, которые %FIXME
%TODO: что есть сейчас

\paragraph{Обучение в кредит}
Это платное обучение, при котором студенту выдаётся кредит на особенных условиях
(с низкой процентной ставкой, без индексации~--- фиксированный на всё время обучения).
%TODO: что есть сейчас

\paragraph{Платное обучение со спонсированием конкретными компаниями}
Любое обучение даётся в качестве подготовки к будущей работе, и было бы логично, 
если компании, нуждающиеся в подготовленных сотрудниках, оплачивали бы их обучение.

\paragraph{Бесплатное обучение с последующим обязательным распределением}
Государство может планировать необходимое число специалистов для каждой сферы деятельности, 
и предоставлять возможность бесплатного обучения лишь необходимому числу людей на конкретных специалистов.
При этом должен заключаться трёхсторонний договор между студентами, государством и предприятиями, 
заинтересованными в специалистах, о том, что предприятия гарантируют трудоустройство специалистов, 
студенты гарантируют прилежное обучение, а государство гарантирует оплату обучения.
Если студент не справляется с обучением, то он обязан будет вернуть деньги за обучение.
Если предприятие не обеспечивает специалиста работой, то оно обязано оплатить его обучение и
предоставить специалисту помощь в трудоустройстве.

\section*{Вывод}

% Bibliography.
\pagebreak
\begin{thebibliography}{99}
  \bibitem{web-akparov}
    Утечка мозгов из России~--- \url{http://akparov.ru/node/54}
\end{thebibliography}

\end{document}
