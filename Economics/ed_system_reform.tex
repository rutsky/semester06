% ed_system_reform.tex
% Term paper on Economics.
% Vladimir Rutsky <altsysrq@gmail.com>
% 23.05.2009

\documentclass[10pt,a4paper,titlepage]{article}

% Spaces after commas.
\frenchspacing
% Minimal carrying number of characters,
\righthyphenmin=2

% Encoding support.
\usepackage{ucs}
\usepackage[utf8x]{inputenc}
\usepackage[T2A]{fontenc}
\usepackage[russian]{babel}
% URL.
\usepackage{url}
% Images.
\usepackage[dvips]{graphicx}
\graphicspath{{./}}

% From K.V.Voroncov Latex in samples, 2005.
\textheight=24cm   % Text height.
\textwidth=16cm    % Text width
\oddsidemargin=0pt % Left side indention.
\topmargin=-1.5cm  % Top side indention.
\parindent=24pt    % Paragraph indent.
\parskip=0pt       % Distance between paragraphs.
\tolerance=2000
%\flushbottom       % Page height aligning.

\title{Необходимые реформы системы образования в России}
\author{Владимир Руцкий, 3057/2}

\begin{document}

\titlepage
\maketitle

% Content.
\section*{Цель работы}
Целью данной работы является анализ текущей системы образования с финансовой стороны, 
насколько вложенные в обучение студентов деньги возвращаются обратно государству, 
и как можно было бы реформировать систему образования для увеличения её ``прибыльности''.

\section*{Анализ проблемы}
На текущий момент в России существует возможность бесплатного обучения на всех уровнях образования 
(школа, ВУЗ, аспирантура).
При этом далеко не всегда человек, получивший образование, 
идёт работать по специальности или вообще остаётся работать в нашей стране, 
и получается, что либо предоставленное бесплатное образование вообще не было нужно человеку, 
либо будет использовано им при работе на другое государство.
Это так называемая ``утечка мозгов''. %TODO: привести статистику

\section*{Возможные решения проблемы}
Как можно обеспечить возврат государству денег, вложенных в обучение человека?

\paragraph{Платное обучение} 
Можно сделать всё обучение, кроме самого базового, платным. 
При этом государство гарантировано не потеряет деньги от вложений в систему образования, 
т.к. практически ничего вкладывать и не будет.
%TODO: что есть сейчас

\paragraph{Обучение в кредит}
Это платное обучение, при котором студенту выдаётся кредит на особенных условиях
(с низкой процентной ставкой, без индексации~--- фиксированный на всё время обучения).
%TODO: что есть сейчас

\paragraph{Платное обучение со спонсированием конкретными компаниями}
Любое обучение даётся в качестве подготовки к будущей работе, и было бы логично, 
если компании, нуждающиеся в подготовленных сотрудниках, оплачивали бы их обучение, хотя бы частично.

\paragraph{Бесплатное обучение с последующим обязательным распределением}
Государство может планировать необходимое число специалистов для каждой сферы деятельности, 
и предоставлять возможность бесплатного обучения лишь необходимому числу людей на конкретных специалистов.
При этом должен заключаться трёхсторонний договор между студентами, государством и предприятиями, 
заинтересованными в специалистах, о том, что предприятия гарантируют трудоустройство специалистов, 
студенты гарантируют прилежное обучение, а государство гарантирует оплату обучения.
Если студент не справляется с обучением, то он обязан будет вернуть деньги за обучение.
Если предприятие не обеспечивает специалиста работой, то оно обязано оплатить его обучение и
предоставить специалисту помощь в трудоустройстве.

\section*{Анализ возможных решений проблем}
Любое образование даётся человеку с целью последующего применения знаний и опыта в какой-то определённой деятельности.
В обязанности государства является продоставление возможности гражданину обучиться тому виду деятельности, 
в которой он может и желает работать. 

Перевод всего образования на платную основу уничтожит всякие надежды получить образование тем, 
у кого нет начального капитала, доставшегося от родителей, 
и, на мой взгляд, только усугубит ситуацию в нашей стране.

Правильным было бы создать общую систему распределения квалифицированных специалистов, 
когда предприятия оценивали бы какой порядок каких специалистов им будет нужен через определённое количество лет,
заявки на этих специалистов распределялись бы между университетами, 
и студенты, поступая на определённую специальность, знали бы, 
что они будут необходимы какому-то конкретному предприятию или компании.

Введение такой системы, вне зависимости от модели оплаты обучения, позволило бы увеличить число людей работающих по специальности.
А введение бесплатного обучения лишь для тех, кто будет работать согласно предписанной специальности на которую они поступали,
заставит людей с ответственностью подходить к выбору будущей специальности и процессу обучения.

Более строгий контроль за тем, каких специалистов и для кого подготавливают университеты также позволит разобраться 
с ``лишними'' университетами, которые не подтверждают свой статус учебного заведения.

\section*{Вывод}
Текущая система образования по большей части предстовляет собой пережиток бывшего СССР, 
методы, которые использовались раньше, не применимы для современной России, 
и система образования нуждается в реформах.

Необходимо изменить модель бесплатного образования так, чтобы люди, получающие его, 
реализовывали вложенные в них деньги путём работы в предписанных их специальностям областях.

% Bibliography.
\pagebreak
\begin{thebibliography}{99}
  \bibitem{web-medvedev-2009-02-02}
    Президент РФ Д.~Медведев о развитии высшего образования и поддержке студенчества~---
      \url{http://www.kremlin.ru/appears/2009/02/02/1039_type207219_212353.shtml}
  \bibitem{web-3}
    Заместитель председателя Комитета образования и науки Государственной думы В.\,Б.~Иванова об образовательных кредитах~---
      \url{http://www.rbcdaily.ru/archive/2004/11/29/31592}
  \bibitem{web-4}
    Министр образования РФ Андрей Фурсенко о реформах системы образования
      \url{http://www.rhr.ru/index/news,9776,0.html}
  \bibitem{web-akparov}
    C.~Миронин, Утечка мозгов: объективный анализ~--- \url{http://www.contr-tv.ru/common/1657/}
\end{thebibliography}

\end{document}
